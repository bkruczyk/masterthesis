\chapter{Algorytmy z wykorzystaniem techniki \emph{rotating calipers}\label{chap:calipers}}
Użyty w rozdziale~\ref{chap:diameter} algorytm można zobrazować jako
zestaw obracających się suwmiarek i jako ogólną metodę można
zastosować przy rozwiązywaniu innych problemów geometrycznych dla
wielokątów wypukłych, pozwalająca na uzyskanie algorytmów działających
w czasie $O(n)$.

\section{Wyznaczanie średnicy}
W tej sekcji zostanie przedstawiony algorytm wyznaczania średnicy z
poprzedniego rozdziału w kontekście techniki \emph{rotating calipers}.

% \subsection{Opis algorytmu}
Niech $P = (p_1, p_2, \ldots, p_n)$ będzie wielokątem
wypukłym. Algorytm z rozdziału \ref{chap:diameter} znajduję wszystkie
pary antypodyczne, a następnie wybiera tę parę, która jest najbardziej
od siebie oddalona, będącą średnicą (na podstawie
tw. \ref{thm:yagbol}). Do znalezienia punktów antypodycznych możemy
użyć suwmiarki składającej się z dwóch równoległych linii
wsparcia. Rozważmy rysunek (). Jako początkową pozycję linii wsparcia
wyznaczmy prostą równoległą do osi $x$ przechodząca przez wierzchołek
$P$ położony najwyżej względem osi $y$ oraz równoległą do niej prostą
przechodząca przez wierzchołek $P$ położony najniżej na osi
$y$. Punkty $p_i$ i $p_j$ są pierwszą znalezioną parą
antypodyczną. Aby znaleźć następną, rozważmy kąty, które tworzy linia
wsparcia przechodząca przez $p_i$ z krawędzią $(p_i, p_{i+1})$ oraz
linia wsparcie przechodząca przez $p_j$ z krawędzią $(p_j, p_{j+1})$.
Jeżeli $\angle{\theta_j} < \angle{\theta_i}$ to obracamy linie
wsparcia o kąt $\theta_j$, w przeciwnym przypadku obracamy linie
wsparcia o kąt $\theta_i$. Punkty $(p_{j+1}, p_i)$ stają się następną
parą antypodyczną. Proces powtarzamy do momentu gdy linie wsparcia
wykonają pełny obrót tj. gdy $p_j = p_{i_0}$ i $p_i = p_{j_0}$.

% rysunek

% pseudokod

\section{Najmniejszy prostokąt zawierający}
Problem najmniejszego prostokąta zawierającego zdefiniujmy
następująco:

\begin{problem}[Najmniejszy prostokąt zawierający]
  Dla wielokąta wypukłego $P$ znaleźć najmniejszy prostokąt taki, żeby
  $P$ był w nim zawarty.
\end{problem}

Rozwiązanie tego problemu ma swoje zastosowanie m. in. w przetwarzaniu
obrazów, grach komputerowych i pewnych algorytmach dotyczących
optymalnego upakowania.

Algorytm z wykorzystaniem suwmiarek opiera się na twierdzeniu:

\begin{twierdzenie}[Freeman-Shapira]
  Najmniejszy prostokąt zawierający wielokąt $P$, posiada bok
  \emph{kolinearny} z bokiem $P$.
\end{twierdzenie}

Jako $L_s(p_i)$ będziemy oznaczać skierowaną linię wsparcia wielokąta
w wierzchołku $p_i$, taką, że $P$ znajduję się po prawej stronie
linii. Rozważmy rysunek (). W pierwszym kroku znajdujemy wierzchołki,
o minimalnych i maksymalnych współrzędnych $x$ i $y$. Oznaczmy te
wierzchołki jako $p_i$, $p_j$, $p_k$, $p_l$. Niech para linii
$L_s(p_j)$ i $L_s(p_l)$ będzie \emph{suwmiarką} pierwszą suwmiarką,
natomiast para $L_s(p_i)$, $L_s(p_k)$ drugą. Analogicznie jak w
algorytmie wyznaczającym średnice, mamy tym razem do rozważenia cztery
kąty: $\angle{\theta_i}, \angle{\theta_j}, \angle{\theta_k},
\angle{\theta_l}$. Obróćmy teraz wszystkie cztery linie wsparcia o
najmniejszy z nich. Wyznaczmy pole prostokąta wyznaczonego przez
punkty przecięć linii wsparcia $L_s(p_i), L_s(p_j), L_s(p_k)$ i
$L_s(p_l)$. Czynność tą powtarzamy z nowo powstałymi kątami
$\angle{\theta_i}, \angle{\theta_j}, \angle{\theta_k},
\angle{\theta_l}$ do czasu gdy rozpatrzone zostaną wszystkie możliwe
prostokąty zawierające tj. obydwie suwmiarki wykonają pełny obrót
wokół $P$. Ze wszystkich wyznaczonych w ten sposób prostokątów
zawierający wybieramy najmniejszy.

\section{Największa odległość\label{sec:max_dist}}
\begin{problem}
  Niech $P = (p_1, p_2, \ldots, p_n)$ oraz $Q = (q_1, q_2, \ldots,
  q_n)$ będą wielokątami wypukłymi. Największą odległość między $P$ i
  $Q$ oznaczamy jako $d_{\max}(P, Q)$ i definiujemy jako: $d_{\max}(P,
  Q) = \max{\{ d(p_i, q_j) \mid i, j = 1, 2, \ldots, n \}}$ gdzie
  $d(p_i, q_j)$ oznacza odległość euklidesową pomiędzy $p_i$ i $q_j$.
\end{problem}

Należy zauważyć, że największa odległość między dwoma wielokątami
wypukłymi nie musi być równa średnicy otoczki wypukłej $P \cup Q$,
więc nie możemy wykorzystać algorytmu wyznaczającego średnicę do
rozwiązania tego problemu.

Rozważmy rysunek (). W pierwszym kroku znajdźmy dwa punkty z których
jeden $p_i$ należy do $P$, a drugi $q_j$ do $Q$, tak aby linie
wsparcia $L_s(p_i)$ i $L_s(p_j)$ były równoległe. Możemy na przykład
wybrać punkty o skrajnym położeniu względem osi $y$. Niech $L_s(p_i)$
i $L_s(q_j)$ będą liniami skierowanymi w przeciwnych
kierunkach. Utworzone w ten sposób linie wsparcia tworzą kąt
$\angle{\theta_i}$ z krawędzią $(p_i, p_{i+1}) \in P$ oraz
kat$\angle{\phi_j}$ z krawędzią $(q_j, q_{j+1}) \in Q$. Wyznaczamy
odległość między $p_i$ i $q_j$. Wybieramy mniejszy z kątów $\phi$ i
$\theta$, a następnie obracamy obydwie linie wsparcia o ten
kąt. Proces powtarzamy do czasu gdy linie wsparcia wykonają pełny
obrót wokół wielokątów. W ostatnim kroku wybieramy najmniejszą z
wyznaczonych odległości między punktami.

\section{Łączenie otoczek wypukłych}
\begin{problem}
  Dla $CH(P)$ i $CH(Q)$ będącymi \emph{otoczkami wypukłymi} zbiorów $P$ i
  $Q$, znaleźć otoczkę wypukłą $CH(CH(P) \cup CH(Q))$.
\end{problem}

W algorytmach typu \emph{dziel i rządź} wyznaczających otoczkę wypukłą
zbioru punktów, końcowym etapem jest łączenie otoczek uzyskanych w
poprzednim kroku, uzyskując w ten sposób coraz większe otoczki, aż do
uzyskania otoczki wypukłej całego zbioru. Jeśli łącznie zostałoby
przeprowadzone w czasie liniowym, złożoność czasowa całego algorytmu
wyniesie $O(n \log n)$[].

Rozważmy dwa wielokąty wypukłe $P$ i $Q$ będącymi otoczkami wypukłymi
pewnych zbiorów punktów (rys). W tym przypadku wyznaczenie otoczki
wypukłej obydwu wielokątów wymaga znalezienia dwóch par wierzchołków
$p_i, p_j$ oraz $q_k, q_l$, takich, że krawędzie $(p_i, q_k)$ i $(p_j,
q_l)$, wraz z łańcuchami wypukłymi $(p_i, p_{i+1}, \ldots, p_j)$ i
$(q_k, q_{k+1}, \ldots, q_l)$ tworzy $CH(P \cup Q)$. Krawędzie $(p_i,
q_k)$ i $(p_j, q_l)$ będziemy nazywać \emph{mostami}, a wierzchołki
tworzące most będziemy nazywać \emph{punktami mostu}.

Algorytm wykorzystujący suwmiarki do wyznaczania mostów opiera się na
następującym twierdzeniu:

\begin{twierdzenie}
\label{thm:bridge}
  Dwa wierzchołki $p_i \in P$ i $q_j \in Q$ są \emph{punktami mostu}
  wtedy i tylko wtedy, gdy tworzą \emph{parę kopodalną} i wierzchołki
  $p_{i-1}, p_{i+1}, q_{j-1}, q_{j+1}$ leżą po tej samej stronie
  prostej $L(p_i, q_j)$.
\end{twierdzenie}

\begin{definicja}
  Para kopodalna
\end{definicja}

Po wyznaczeniu pozycji suwmiarek, analogicznie jak w algorytmie
\ref{sec:max_dist}, skierowanych w przeciwnych kierunkach, zaczynamy
obracać je wokół wielokątów. Podobnie jak w poprzednich problemach
rozważamy kąt jaki tworzy skierowana linia wsparcia wielokąta $P$ w
punkcie $p_i$ z krawędzią $(p_{i+1}, p_{i+2})$ oraz linia wsparcia
wielokąta $Q$ z krawędzią $(q_{j+1}, q_{j+2})$. Obydwie suwmiarki
obracamy o mniejszy z tych kątów. Przed każdym kolejnym obrotem
sprawdzamy czy aktualna para wierzchołków $p_i, q_j$ spełnia warunek
zawarty w tw.\ref{thm:bridge}. Ponieważ przy każdym obrocie suwmiarek
generowana jest nowa para kopodalną, pozostaje sprawdzić czy
wierzchołki $p_{i-1}, p_{i+1}, q_{j-1}, q_{j+1}$ leżą po tej samej
stronie prostej $L(p_i, q_j)$. Algorytm znajdowania mostów w przypadku
łączenia otoczek rozdzielnych wypukłych, tak jak ma to miejsce w
przypadku algorytmu dziel i rządź wyznaczania otoczki wypukłej zbioru
punktów, można zakończyć po znalezieniu obydwu mostów.

Niech liczba wierzchołków $P$ wynosi $n$, a liczba wierzchołków $Q$
wynosi $m$. Par ko-podalnych jest $k < n + m$. Sprawdzenie czy para
ko-podalna jest mostem odpobywa się w czasie $O(1)$, stąd złożoność
czasowa algorytmu wyznaczającego mosty jest liniowa i wynosi $O(k)$.

%%% Local Variables:
%%% mode: latex
%%% TeX-master: "masterthesis"
%%% TeX-engine: xetex
%%% End:
