\chapter{Wyznaczanie średnicy\label{chap:diameter}}
Problem średnicy dowolnego zbioru punktów można przedstawić
następująco: \emph{mając danych $n$ punktów na płaszczyźnie, znaleźć
  dwa punkty najbardziej odległe od siebie}. Problem można rozwiązać
metodą naiwną wyznaczając odległość wszystkich $n(n-1)/2$ par punktów
i wybierając największą z nich. Naturalnym jest jednak szukanie
bardziej efektywnych rozwiązań.

Aby uniknąć badania wszystkich par punktów, można skorzystać z
twierdzenia~\ref{thm:hy61}: wystarczy najpierw wyznaczyć otoczkę
wypukłą zbioru punktów w czasie
liniowo-logarytmicznym~\cite{Graham72}, a następnie wyznaczyć jej
średnicę.

\begin{twierdzenie}[Hocking-Young 1961\label{thm:hy61}]
  Średnica zbioru jest równa średnicy jego otoczki wypukłej.
\end{twierdzenie}

Wiemy, że zbiór wierzchołków wielokąta wypukłego jest jego otoczką
wypukłą, dlatego zdefiniujmy następujący problem.

\begin{problem}[Średnica wielokąta wypukłego]
  Jeśli dany jest wielokąt wypukły, znaleźć jego średnicę,
  tj.\ największą odległość pomiędzy dowolną parą jego wierzchołków.
\end{problem}

Zauważmy, że średnica wielokąta wypukłego jest w istocie najdłuższym z
odcinków łączących dwa jego wierzchołki. Weźmy dowolny punkt $o$
należący do wielokąta wypukłego $P$. Ze względu na wypukłość, punkt
$p_i$ od niego najdalszy musi być jednym z wierzchołków wielokąta $P$
--- rozważając kolejne punkty na krawędzi $(p_{i-1}, p_i)$ zgodnie z
ruchem wskazówek zegara ich odległość od $o$ zwiększa się osiągając
maksimum w punkcie $q_i$, maleje z kolei wzdłuż krawędzi $(p_i,
p_{i+1})$.

Kluczowym dla algorytmu korzystającego z wypukłości wielokąta przy
wyznaczaniu średnicy jest następujące twierdzenie.

\begin{twierdzenie}[Yaglom-Boltyanskii 1961]
\label{thm:yagbol}
  Średnica wielokąta wypukłego jest największą odległością między parą
  równoległych prostych wspierających.
\end{twierdzenie}

Proste wspierające nie mogą przechodzić przez każdą parę punktów. Na
przykład żadne, dwie różne proste wpierające przechodzące przez $p_3$
i $p_6$ wielokąta z rysunku~\ref{fig:antipodal} nie mogą być
równoległe, więc para $(p_{3},p_{6})$ nie może definiować
średnicy. Przypomnijmy --- para punktów, przez które mogą przechodzić
proste wspierające, jest nazywana punktami antypodycznymi. Z
twierdzenia~\ref{thm:yagbol} wynika, że szukając średnicy nie musimy
badać wszystkich par punktów, a jedynie pary antypodyczne.

\begin{figure}[htp]
  \centering
  \begin{tikzpicture}
    \coordinate (p3) at (5, 3);
    \coordinate (p4) at (4, 4.5);
    \coordinate (p5) at (2, 5);
    \coordinate (p6) at (0, 3);
    \coordinate (p0) at (-1, 0);
    \coordinate (p1) at (3, 0);
    \coordinate (p2) at (5, 1);

    \draw (p0) -- (p1) -- (p2) -- (p3) -- (p4) --  (p5) -- (p6) -- cycle;

    \node [anchor=center,circle,draw,fill,inner sep=1pt,
    label={250:$p_0$}] at (p0) {};

    \node [anchor=center,circle,draw,fill,inner sep=1pt,
    label={below:$p_1$}] at (p1) {};

    \node [anchor=center,circle,draw,fill,inner
    sep=1pt,label={right:$p_2$}] at (p2) {};

    \node [anchor=center,circle,draw,fill,inner
    sep=1pt,label={right:$p_3$}] at (p3) {};

    \node [anchor=center,circle,draw,fill,inner
    sep=1pt,label={above:$p_4$}] at (p4) {};

    \node [anchor=center,circle,draw,fill,inner
    sep=1pt,label={above:$p_5$}] at (p5) {};

    \node [anchor=center,circle,draw,fill,inner
    sep=1pt,label={left:$p_6$}] at (p6) {};

  \end{tikzpicture}
  \caption{Para wierzchołków $(p_3,p_6)$ nie może definiować
    średnicy.\label{fig:antipodal}}
\end{figure}

Przedstawiony zostanie teraz sposób znajdowania par punktów
antypodycznych. Rozważmy wielokąt przedstawiony na
rysunku~\ref{fig:antipodal}. Wybieramy dowolny wierzchołek $p_i$ i
przechodzimy w kierunku przeciwnym do ruchu wskazówek zegara po brzegu
wielokąta tak długo, aż dojdziemy do wierzchołka $q_r$, który jest
najdalej od prostej zawierającej krawędź $p_{i}p_{i-1}$
(rysunek~\ref{fig:diameter}). W taki sam sposób wyznaczamy wierzchołek
$q_l$ jako najdalszy od $p_{i}p_{i+1}$ przechodząc po brzegu wielokąta
zgodnie z ruchem wskazówek zegara. Łańcuch $C(p_i)$ wierzchołków od
$q_r$ do $q_l$, z nimi samymi włącznie, tworzy zbiór wierzchołków, z
których każdy tworzy parę antypodyczną z $p_i$. Możemy zauważyć, że
($p_{i},p_{r}$) jest parą antypodyczną wyłącznie wtedy, gdy istnieje
prosta przecinająca $\angle \alpha_i$ i $\angle \alpha_s$
(rysunek~\ref{fig:diameter}). Ze względu na to, że $C(p_i)$ jest
łańcuchem wypukłym, każdy wierzchołek $p_s$ należący do $C(p_i)$ jest
antypodyczny do $p_i$, a żaden wierzchołek nienależący do $C(p_i)$
takim nie jest. Fakt ten umożliwia konstrukcję algorytmu pozwalającego
efektywnie uzyskać wszystkie pary antypodyczne.

\begin{figure}[htp]
  \centering
  \begin{tikzpicture}
    \coordinate (p3) at (5, 3);
    \coordinate (p4) at (4, 4.5);
    \coordinate (p5) at (2, 5);
    \coordinate (p6) at (0, 3);
    \coordinate (p0) at (-1, 0);
    \coordinate (p1) at (3, 0);
    \coordinate (p2) at (5, 1);

    \draw (p0) -- (p1) -- (p2) -- (p3) -- (p4) --  (p5) -- (p6) --
    cycle;

    \draw (p0) -- (p1);

    \node [anchor=center,circle,draw,fill,inner sep=1pt,
    label={}] at (p0) {};

    \node [anchor=center,circle,draw,fill,inner sep=1pt,
    label={}] at (p1) {};

    \node [anchor=center,circle,draw,fill,inner
    sep=1pt,label={}] at (p2) {};

    \node [anchor=center,circle,draw,fill,inner
    sep=1pt,label={}] at (p3) {};

    \node [anchor=center,circle,draw,fill,inner
    sep=1pt,label={}] at (p4) {};

    \node [anchor=center,circle,draw,fill,inner
    sep=1pt,label={}] at (p5) {};

    \node [anchor=center,circle,draw,fill,inner sep=1pt,
    label={below:$p_{i+1}$}] at (p1) {};

    \node [anchor=center,circle,draw,fill,inner sep=1pt,
    label={below:$p_{i}$}] at (p0) {};

    \node [anchor=center,circle,draw,fill,inner sep=1pt,
    label={120:$p_{i-1}$}] at (p6) {};

    \node [anchor=center,circle,draw,fill,inner sep=1pt,
    label={right:$q_r$}] at (p2) {};

    \node [anchor=center,circle,draw,fill,inner sep=1pt,
    label={above:$q_l$}] at (p5) {};

    \draw [dashed,blue,shorten >= -3cm] (p0) -- (p6);
    \draw [dashed,blue,shorten >= -5cm] (p0) -- (p1);

    \draw pic["$\alpha_i$",draw=blue, dashed, angle eccentricity=1.5, angle
    radius=0.75cm]{angle=p6--p0--p1};

    \draw [dashed,red,shorten >= -3cm] (p4) -- (p5);
    \draw [dashed,red,shorten >= -4.5cm] (p4) -- (p3);

    \draw pic["$\alpha_s$",draw=red, dashed, angle eccentricity=1.5, angle
    radius=0.75cm]{angle=p3--p4--p5};

    \coordinate (x1) at (intersection of p0--p6 and p4--p5);
    \coordinate (x2) at (intersection of p0--p1 and p4--p3);

    \draw [shorten >= -1cm, shorten <= -1cm] (x1) -- (x2);

        \node [label={[label distance=3cm]327:$l$}] at (x1) {};
  \end{tikzpicture}
  \caption{Prosta $l$ przecina $\alpha_i$ i $\alpha_s$.\label{fig:diameter}}
\end{figure}

Wejściem dla algorytmu jest wielokąt zadany jako lista wierzchołków ze
wskaźnikiem na swój następnik w kolejności przeciwnej do ruchu
wskazówek zegara --- następnikiem wierzchołka $p_{n-1}$ jest
$p_0$. Dla czytelności algorytmu przyjmijmy także, że żadna para boków
tego wielokąta nie jest równoległa. Do określenia, który punkt leży
najdalej od odcinka $p_{i}p_{i+1}$, korzystamy z funkcji ,,podwojonego
pola ze znakiem trójkąta'' $p_{0}p_{i+1}p$, którą przedstawiono w
rozdziale~\ref{chap:pojecia} do wyznaczenia skrętności kąta. W
pierwszej części algorytmu w pętli \textbf{while} szukamy pierwszego
najdalszego wierzchołka i oznaczamy go jako $q_0$. Następnie
wyznaczamy łańcuch $C(p_i)$, używając wskaźników $p$ i $q$
poruszających się w kierunku odwrotnym do ruchu wskazówek zegara,
zaczynając odpowiednio od $p_0$ i $q_0$ do momentu, aż wskaźnik $q$
wróci do pozycji $p_0$. Po tym kroku zostaną wypisane wszystkie pary
antypodyczne. Znalezienie wierzchołka $q_0$ zajmuje w pesymistycznym
przypadku $n-1$ kroków, natomiast przejście po łańuchach $p_{0},
\ldots, q_{0}$ i $q_{0}, \ldots, p_{0}$ łącznie $n$ kroków. Mamy więc
do czynienia ze złożonością czasową rzędu $O(n)$.

% Specjalną sytuacją jest przypadek, gdy dwie krawędzie są równoległe
% (wtedy pole ze znakiem trójkąta jest równe dla dwóch wierzchołków
% tworzących krawędź równoległą), która wymaga od nas dodatkowego kroku
% przy każdym wierzchołku (wiersze 15--19).

% Poprawność algorytmu wynika z tego, że krawędzi równoległych jest co
% najwyżej $N/2$, więc łącznie algorytm nie wykonuje więcej niż $3N$
% kroków. Mamy więc do czynienia ze złożonością czasową rzędu $O(n)$.

Fakt, że punkt antypodyczny dla punktu $p_{i+1}$ nie może być bliższy
niż wierzchołek, który jest najdalej od $p_i$, wynika z wypukłością
wielokąta --- dla dowolnego wierzchołka $p$ można wyznaczyć taki
wierzchołek $q$, że odległość kolejnych wierzchołków od $p$ do $q$
jest nierosnąca, zaś od $q$ do $p$ --- niemalejąca.

\begin{figure}[htp]

  \begin{algorithmic}[1]
    \Procedure{Antipodal Pairs}{}

    \State $p \gets p_{n-1}$
    \State $q \gets NEXT[p]$

    \State

    \While {$NEXT[q]$ jest dalej od $(p, NEXT[p])$ niż $q$}
        \State {\emph{/* znajdźmy wierzchołek najdalszy od $(p, NEXT[p])$ */}}
        \State{$q \gets NEXT[p]$}

        \State

        \State {\emph{/* oznaczmy go jako $q_0$ */}}
        \State {$q_0 \gets q$}

        \State

        \While {$q \neq p_0$}
                \State {$p \gets NEXT[p]$}

                \State $print (p, q)$

                \State

                \While {$NEXT[q]$ jest dalej od odcinka $(p, NEXT[p])$ niż $q$}
                        \State $q \gets NEXT[q]$

                        \State

                        \State {\emph{/* jeżeli nie wróciliśmy do pozycji początkowej */ }}
                        \If {$(p, q) \neq (q_0,p_0)$}
                                \State $print (p, q)$
                        \EndIf
                \EndWhile
         \EndWhile
     \EndWhile
\EndProcedure

  \end{algorithmic}
  \caption{Algorytm wypisujący wszystkie pary
    antypodalne.\label{alg:antipodal}}
\end{figure}

\begin{table}[htb]
  \centering

  \begin{tabular}{llll}
    \toprule
    $p$ & $q$ & para antypodyczna & krok \\
    \midrule
    $p_6$ & & & 1 \\
    \midrule
    & $p_0$ & & 3 \\
    \midrule
    & $p_1$ & & 5 \\
    \midrule
    & $p_2$ & & 5 \\
    \midrule
    $p_0$& & & 13 \\
    \midrule
    & & $(p_0, p_2)$ & 14 \\
    \midrule
    & $p_3$ & & 17 \\
    \midrule
    & & $(p_0, p_3)$ & 21 \\
    \midrule
    & $p_4$ & & 17 \\
    \midrule
    & & $(p_0, p_4)$ & 21 \\
    \midrule
    & $p_5$ & & 17 \\
    \midrule
    & & $(p_0, p_5)$ & 21 \\
    \midrule
    $p_1$ & & & 13 \\
    \midrule
    & & $(p_1, p_5)$ & 14 \\
    \midrule
    $p_2$ & & & 13 \\
    \midrule
    & & $(p_2, p_5)$ & 14 \\
     \midrule
    & $p_6$ & & 17 \\
    \midrule
    & & $(p_2, p_6)$ & 21 \\
     \midrule
    & $p_0$ & & 17 \\
    \bottomrule
  \end{tabular}

  \caption{Ilustracja działania algorytmu  \textsc{Antipodal Pairs} dla
    wielokąta z rysunku~\ref{fig:antipodal}.}
\end{table}

%%% Local Variables:
%%% mode: latex
%%% TeX-master: "masterthesis"
%%% TeX-engine: xetex
%%% End:
