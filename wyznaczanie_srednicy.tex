\chapter{Wyznaczanie średnicy}
Jednym z zagadnień, w którym kluczowym jest wyznaczanie średnicy
zbioru punktów jest problem klastrowania. Klastrowanie zbioru to
grupowanie jego elementów o podobnych właściwościach. Można zauważyć,
że elementy klastra o mniejszej średnicy są ze sobą ściślej powiązane
niż w klastrze dużym. Przez wielkość klastra mamy na myśli jego
\emph{rozrzucenie}, czyli największą odległość między dwoma punktami
klastra. Jeśli mamy więc zbiór punktów podzielić na $K$ klastrów,
pożądany jest taki podział aby największa średnica klastra była
możliwie mała.

%%% Local Variables:
%%% mode: latex
%%% TeX-master: "masterthesis"
%%% TeX-engine: xetex
%%% End:
