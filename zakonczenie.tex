\summary{

  Celem pracy było pokazanie, że własność wypukłości wpływa na
  efektywność i prostotę rozwiązań problemów geometrycznych związanych
  z wielokątami.

  Przedstawione problemy oraz ich rozwiązania wskazują, że dzięki
  znajomości faktu o tym czy dany wielokąt jest wypukły, możemy
  formułować pewne niezmienniki i wykorzystywać lokalne w danym kroku
  algorytmu własności charakterystyczne dla wielokątów wypukłych,
  które pozwalają na podanie prostszych algorytmów oraz na łatwiejsze
  wykazanie ich poprawności.

  Przy lokalizacji punktu był to fakt, iż wierzchołki wielokąta
  wypukłego występują w kolejności kątowej, a przy wyznaczaniu
  średnicy korzystaliśmy z faktu istnienia liniowej ilości par
  antypodycznych w wielokącie, które natomiast zostało szerzej
  wykorzystane w szeregu algorytmów z rozdziału piątego. Przy badaniu
  przecięć wielokątów metodą wymiatania Shamosa-Hoeya było to
  istnienie stałej liczby przecięć krawędzi w każdym pasie podziału
  płaszczyzny; metoda O'Rourke i in.\ wykorzystywała fakt istnienia
  sierpów w przypadku przecięcia dwóch wypukłych wielokątów, natomiast
  metoda Touissanta stosowała do tego celu pojęcie kieszeni. W
  algorytmie dla problemu zawierania wielokątów kluczowym było pojęcie
  wierzchołka krytycznego z wielokąta zawieranego dla krawędzi z
  wielokąta zawierającego.

  Prace~\cite{Prep85} oraz~\cite{Chazelle80} wskazują na to, że
  własność wypukłości jest pomocna również w rozwiązaniach problemów w
  przestrzeniach wyższych od $\mathbb{R}^2$.

}

%%% Local Variables:
%%% mode: latex
%%% TeX-master: "masterthesis"
%%% TeX-engine: xetex
%%% End:
