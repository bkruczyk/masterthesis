\documentclass[brudnopis]{xelatex-mgr/xmgr}

\usepackage{amsmath}
\usepackage{amssymb}
\usepackage{blindtext}
\usepackage{tikz}
\usepackage{amsthm}
\usepackage{booktabs}
\usepackage{algpseudocode}
\usepackage[boxed]{algorithm}

\usetikzlibrary{calc}
\usetikzlibrary{intersections}
\usetikzlibrary{patterns}
\usetikzlibrary{quotes}
\usetikzlibrary{angles}
\tikzset{>=latex}

% \defaultfontfeatures{Scale=MatchLowercase}
\setmainfont[Numbers=OldStyle,Mapping=tex-text]{Minion Pro}
\setsansfont[Numbers=OldStyle,Mapping=tex-text]{Myriad Pro}
\setmonofont[Scale=0.75]{Monaco}

\wersja{wersja wstępna [\ymdtoday]}

\author{Bartłomiej Kruczyk}
\nralbumu{213603}
\email{bartlomiej.kruczyk@gmail.com}

\title{Efektywne algorytmy dla wielokątów wypukłych}
\date{2014}
\miejsce{Gdańsk}

\opiekun{dr hab. Paweł Żyliński}

\newtheorem{twierdzenie}{Twierdzenie}[chapter]
\newtheorem{lemat}{Lemat}[section]

\theoremstyle{definition}

\newtheorem{problem}{Problem}[chapter]
\newtheorem{definicja}{Definicja}[chapter]

\setcounter{tocdepth}{1}

\usepackage{etoolbox}
\makeatletter
\patchcmd{\l@chapter}{1.0em}{0.8em}{}{}
\makeatother

\makeatletter
\renewcommand{\ALG@name}{Listing}
\renewcommand{\listalgorithmname}{Spis listingów}
\makeatother

\includeonly{
  frontmatter
  ,wprowadzenie
  ,podstawowe_pojecia
  ,lokalizacja_punktu
  ,wyznaczanie_srednicy
  ,przeciecia_wielokatow
  ,rotating_calipers
  ,zawieranie_wielokatow
  ,zakonczenie
  ,bibliography
  ,endmatter
}

\begin{document}

\begin{abstract}
  Celem niniejszej pracy jest przedstawienie szeregu problemów
  geometrycznych dotyczących wielokątów wypukłych oraz ich rozwiązań
  korzystających z własności wypukłości, prowadząc do podania
  algorytmów działających w czasie liniowym. Opis rozwiązań zawiera
  analizę działania wraz ze złożonością czasową i dowodem poprawności
  wybranych algorytmów.
\end{abstract}

\keywords{
  wielokąty wypukłe,
  lokalizacja punktu,
  średnica wielokąta,
  przecięcie wielokątów,
  rotating calipers,
  zawieranie wielokątów
}

\maketitle

%%% Local Variables:
%%% mode: latex
%%% TeX-master: "masterthesis"
%%% TeX-engine: xetex
%%% End:

\introduction{
% geometria obliczeniowa i problemy geometryczne, zastosowania
% wykorzystanie wypuklosci w problemach geometria
% niniejsza praca

  Problemy geometryczne, ze względu na swoje praktyczne zastosowania,
  są jednym z najstarszym przedmiotów rozważań algorytmicznych. Wraz z
  rozwojem technologi, narodziło się pojęcie geometrii obliczeniowej
  --- dyscypliny rozwiązującej problemy geometryczne z wykorzystaniem
  komputerów w modelu obliczeń. Wiele z rozwiązań znalazło użytek w
  systemach informacji geograficznej, robotyce, grafice komputerowej
  czy też w narzędziach CAD/CAM wspomagających pracę projektantów i
  architektów.

  Okazuje się, że posiadanie przez wielokąt własności wypukłości,
  pozwala na podanie szybszych lub prostszych rozwiązań takiego samego
  problemu dla dowolnego wielokąta. Celem niniejszej pracy jest
  przedstawienie szeregu problemów dotyczących wielokątów wypukłych
  oraz ich rozwiązań.

  W rozdziale pierwszym...

  W rozdziale drugim...

  W rozdziale
}

%%% Local Variables:
%%% mode: latex
%%% TeX-master: "masterthesis"
%%% TeX-engine: xetex
%%% End:

\chapter{Podstawowe pojęcia}\label{chap:pojecia}
W tym rozdziale zostaną przybliżone podstawowe pojęcia związane
z geometrią obliczeniową i problemami dotyczącymi wielokątów
wypukłych.

\begin{definicja}
  \emph{Wielokątem prostym} nazywamy taki wielokąt, że jedyne punkty
  płaszczyzny należące jednocześnie do dwóch jego krawędzi są jego
  wierzchołkami.
\end{definicja}

\begin{definicja}
  \emph{Wielokątem wypukłym} nazywamy taki wielokąt prosty którego
  wnętrze jest \emph{zbiorem wypukłym}, tzn.\ wszystkie punkty
  należące do odcinka łączącego dwa dowolne punkty ze zbioru wypukłego
  należą do tego zbioru.
\end{definicja}

\begin{figure}[htb]
  \centering
  \includegraphics{img/nonconvex}
  \caption{Figura niebędąca wielokątem wypukłym.}
\end{figure}

\begin{definicja}
  \emph{Średnicą zbioru punktów} nazywamy największą odległość
  pomiędzy dwoma punktami należącymi do zbioru.
\end{definicja}

\begin{definicja}
  Niech $p_{1}=(x_{1},y_{1})$, $p_{2}=(x_{2},y_{2})$,
  $p_{3}=(x_{3},y_{3})$ będą punktami na płaszczyźnie
  $R^2$. \emph{Wyznacznikiem} współrzędnych tych punktów nazywamy
  liczbę

  \begin{center}
    \begin{math}
      X(p_1, p_2, p_3) =
      \begin{vmatrix}
        x_1 & y_1 & 1 \\
        x_2 & y_2 & 1 \\
        x_3 & y_3 & 1
      \end{vmatrix}
    \end{math}
  \end{center}

  Kąt $\angle p_{1},p_{2},p_{3}$ nazywamy \emph{lewoskrętnym}, jeżeli
  wyznacznik $X(p_1, p_2, p_3)$ jest dodatni, w przeciwnym przypadku
  mówimy, że kąt jest \emph{prawoskrętny}.
\end{definicja}

\begin{definicja}
  \emph{Prostymi wspierającymi} dla wielokąta nazywamy takie proste,
  które przechodząc przez wierzchołek wielkąta nie przecinają jego
  wnętrza.
\end{definicja}

\begin{definicja}
  Mówimy, że para wierzchołków tworzy \emph{punkty antypodyczne},
  jeżeli można poprowadzić przez te wierzchołki przynajmniej dwie
  różne równoległe proste wspierające.
\end{definicja}

\begin{definicja}\label{def:bigo}
  Mówimy, że funkcja $f\colon N \to R$ jest co najwyżej rzędu $g$
  (jest ograniczona przez funkcję $g\colon N \to R$), gdy istnieją
  takie stałe $n_0 > 0$ oraz $c > 0$, takie że:

  \begin{center}
    $\forall n \geq n_0 : f(n) \leq c \cdot g(n)$
  \end{center}
\end{definicja}

W niniejszej pracy do opisu wydajności algorytmów będziemy korzystać
z \emph{notacji wielkiego O}. Jest to notacja asymptotycznego tempa
wzrostu wartości funkcji względem jej argumentów, w algorytmice
stosowana do charakterystyki złożoności obliczeniowej algorytmów
opisując ilość potrzebnych zasobów (czasu lub pamięci) w stosunku
do rozmiaru danych wejściowych. Zgodnie z notacją, złożoność czasową
funkcji z definicji \ref{def:bigo} zapisalibyśmy jako $O(g(n))$.

\begin{table}[htb]
  \centering

  \begin{tabular}{ll}
    \toprule
    $O(1)$ & stała \\
    \midrule
    $O(\log n)$ & logarytmiczna \\
    \midrule
    $O(n)$ & liniowa \\
    \midrule
    $O(n \log n)$ & liniowo-logarytmiczna \\
    \midrule
    $O(n^2)$ & kwadratowa \\
    \midrule
    $O(n^c)$ & wielomianowa \\
    \midrule
    $O(c^n)$ & wykładnicza \\
    \midrule
    $O(n!)$ & ograniczona przez silnię \\
    \bottomrule
  \end{tabular}

  \caption{Najczęściej wyróżniane rzędy żłożoności obliczeniowej,
    podane według rosnącej złożoności.}
\end{table}

W tej pracy \emph{efektywnym} algorytmem dla wielokąta wypukłego będziemy
nazywać algorytm o mniejszej złożoności obliczeniowej niż najlepszy
znany algorytm uogólniony dla wielokąta prostego dla tego samego
problemu.

%%% Local Variables:
%%% mode: latex
%%% TeX-master: "masterthesis"
%%% TeX-engine: xetex
%%% End:

\newcommand{\convexA}{
  \coordinate (p0) at (4.5,2);
  \coordinate (p1) at (4,3.25);
  \coordinate (p2) at (2,4);
  \coordinate (p3) at (1,3.5);
  \coordinate (p4) at (0,2);
  \coordinate (p5) at (2,0);
  \coordinate (p6) at (3.75,1);

  \draw (p0) -- (p1) -- (p2) -- (p3) -- (p4) -- (p5) -- (p6) -- cycle;
}

\newcommand{\convexB}{
  \coordinate (p0) at (3.5,     4);
  \coordinate (p1) at (1.5,     4.5);
  \coordinate (p2) at (0,       3.5);
  \coordinate (p3) at (-0.5,    1.5);
  \coordinate (p4) at (1.75,    0);
  \coordinate (p5) at (4,       1);

  \draw (p0) -- (p1) -- (p2) -- (p3) -- (p4) -- (p5) -- cycle;
}

\chapter{Lokalizacja punktu}
Problem lokalizacji punktu sformułowany jest następująco.

\begin{problem}[Lokalizacja punktu]
  Jeśli dany jest wielokąt prosty $P$ i punkt $z$ na płaszczyźnie,
  sprawdzić czy punkt $z$ należy do wnętrza $P$.
\end{problem}

Problem ten można rozwiązać w czasie $O(n)$ bez przetwarzania
wstępnego. Rozważmy poziomą prostą $l$ przechodząca przez punkt
$z$. Musimy rozważyć kilka przypadków.

\begin{itemize}
\item Jeśli $l$ nie przecina $P$, to $z$ leży na zewnątrz wielokąta.
  \begin{figure}[htp]
    \centering
    \begin{tikzpicture}
      \convexA

      \node [anchor=center,circle,draw,fill,inner sep=0.5pt,
      label={right:$p_0$}] at (p0) {};

      \node [anchor=center,circle,draw,fill,inner sep=0.5pt,
      label={45:$p_1$}] at (p1) {};

      \node [anchor=center,circle,draw,fill,inner
      sep=0.5pt,label={above:$p_2$}] at (p2) {};

      \node [anchor=center,circle,draw,fill,inner
      sep=0.5pt,label={135:$p_3$}] at (p3) {};

      \node [anchor=center,circle,draw,fill,inner
      sep=0.5pt,label={left:$p_4$}] at (p4) {};

      \node [anchor=center,circle,draw,fill,inner
      sep=0.5pt,label={below:$p_5$}] at (p5) {};

      \node [anchor=center,circle,draw,fill,inner
      sep=0.5pt,label={315:$p_6$}] at (p6) {};

      \coordinate (z) at (3,-1);
      \coordinate (l) at (-1,-1);

      \draw [shorten >=-3cm, shorten <=-1cm] (l) -- (z);

      \node [anchor=center,circle,draw,fill,inner
      sep=0.5pt,label={below:$z$}] at (z) {};

      \node [label={[label distance=-0.2cm]above:$l$}] at (l) {};
    \end{tikzpicture}
    \caption{}
  \end{figure}

\item W przypadku, gdy prosta $l$ przecina $P$ i nie przechodzi przez
  żaden z wierzchołków $P$, oznaczmy jako $p_l$ liczbę przecięć $l$
  z wielokątem $P$. Wiemy, że lewy koniec $l$ leży na zewnątrz $P$,
  więc przesuwając się po prostej $l$ w prawo w kierunku $z$
  i przecinając brzeg wielokąta, przechodzimy do jego wnętrza. Przy
  następnym przecięciu z brzegiem $P$ znowu znajdziemy się na zewnątrz
  itd. Widzimy stąd, że $z$ jest na zewnątrz wielokąta $P$ dokładnie
  wtedy, gdy $p_l$ jest parzyste.

  \begin{figure}[htp]
    \centering
    \begin{tikzpicture}
      \convexA

      \node [anchor=center,circle,draw,fill,inner sep=0.5pt,
      label={right:$p_0$}] at (p0) {};

      \node [anchor=center,circle,draw,fill,inner sep=0.5pt,
      label={45:$p_1$}] at (p1) {};

      \node [anchor=center,circle,draw,fill,inner
      sep=0.5pt,label={above:$p_2$}] at (p2) {};

      \node [anchor=center,circle,draw,fill,inner
      sep=0.5pt,label={135:$p_3$}] at (p3) {};

      \node [anchor=center,circle,draw,fill,inner
      sep=0.5pt,label={left:$p_4$}] at (p4) {};

      \node [anchor=center,circle,draw,fill,inner
      sep=0.5pt,label={below:$p_5$}] at (p5) {};

      \node [anchor=center,circle,draw,fill,inner
      sep=0.5pt,label={315:$p_6$}] at (p6) {};

      \coordinate (z) at (3,2.5);
      \coordinate (l) at (-1,2.5);

      \draw [shorten >=-3cm, shorten <=-1cm] (l) -- (z);

      \node [anchor=center,circle,draw,fill,inner
      sep=0.5pt,label={below:$z$}] at (z) {};
      \node [label={[label distance=-0.2cm]above:$l$}] at (l) {};
    \end{tikzpicture}
    \caption{}
  \end{figure}

\item Specjalnym przypadkiem jest ten, w którym $l$ przechodzi przez
  jeden lub dwa wierzchołki. W algorytmie rozważamy przecięcia $l$ z
  krawędziami $P$, więc musimy się zabezpieczyć przed sytuacją, w
  której policzylibyśmy przecięcie brzegu $P$ dwukrotnie. Stosujemy tu
  zasadę, że $p_l$ nie zostanie zwiększone dla krawędzi, której jeden
  z wierzchołków znajduje się powyżej prostej $l$. W przypadku gdy
  prosta $l$ przechodzi przez oba wierzchołki krawędzi wielokąta,
  przecięcie nie jest zliczane.

  \begin{figure}[htp]
    \centering
    \begin{tikzpicture}
      \convexA

      \node [anchor=center,circle,draw,fill,inner sep=0.5pt,
      label={-10:$p_0$}] at (p0) {};

      \node [anchor=center,circle,draw,fill,inner sep=0.5pt,
      label={45:$p_1$}] at (p1) {};

      \node [anchor=center,circle,draw,fill,inner
      sep=0.5pt,label={above:$p_2$}] at (p2) {};

      \node [anchor=center,circle,draw,fill,inner
      sep=0.5pt,label={135:$p_3$}] at (p3) {};

      \node [anchor=center,circle,draw,fill,inner
      sep=0.5pt,label={235:$p_4$}] at (p4) {};

      \node [anchor=center,circle,draw,fill,inner
      sep=0.5pt,label={below:$p_5$}] at (p5) {};

      \node [anchor=center,circle,draw,fill,inner
      sep=0.5pt,label={315:$p_6$}] at (p6) {};

      \coordinate (z) at (3,2);
      \coordinate (l) at (-1,2);

      \draw [shorten >=-3cm, shorten <=-1cm] (l) -- (z);

      \node [anchor=center,circle,draw,fill,inner
      sep=0.5pt,label={below:$z$}] at (z) {};
      \node [label={[label distance=-0.2cm]above:$l$}] at (l) {};
    \end{tikzpicture}
    \caption{Przecięcie zostanie policzone dla krawędzi $p_{4}p_{5}$
      oraz $p_{6}p_{0}$.}
  \end{figure}

\end{itemize}

Ze względu na to, że każdą krawędź $P$ sprawdzamy tylko raz
pod względem przecięcia z $l$, mamy do czynienia ze złożonością
liniową zależną od liczby krawędzi wielokąta.

Algorytm dla wielokąta wypukłego wymaga pewnego przetwarzania
wstępnego, ale jest użyteczny w przypadku wykonywania wielokrotnych
zapytań o przynależność punktu do danego wielokąta. Weźmy punkt $q$
leżący wewnątrz wielokąta wypukłego $P$, może to być np.\ środek
ciężkości trójkąta wyznaczony przez trzy dowolne jego wierzchołki,
a następnie poprowadźmy $N$ półprostych z $q$ przechodzących przez
wierzchołki $P$. Płaszczyzna wraz z wielokątem zostanie podzielona
na $N$ części, które nazwiemy \emph{klinami}, z których każdy jest
podzielony przez krawędź $P$ na dwie częsci --- jest to wnętrze
i zewnętrze wielokąta.

\begin{figure}[htp]
  \centering
  \begin{tikzpicture}
    \convexA

    \node [anchor=center,circle,draw,fill,inner sep=0.5pt,
    label={10:$p_0$}] at (p0) {};

    \node [anchor=center,circle,draw,fill,inner sep=0.5pt,
    label={90:$p_1$}] at (p1) {};

    \node [anchor=center,circle,draw,fill,inner
    sep=0.5pt,label={100:$p_2$}] at (p2) {};

    \node [anchor=center,circle,draw,fill,inner
    sep=0.5pt,label={180:$p_3$}] at (p3) {};

    \node [anchor=center,circle,draw,fill,inner
    sep=0.5pt,label={[label distance=0.2cm]-90:$p_4$}] at (p4) {};

    \node [anchor=center,circle,draw,fill,inner
    sep=0.5pt,label={-45:$p_5$}] at (p5) {};

    \node [anchor=center,circle,draw,fill,inner
    sep=0.5pt,label={[label distance=0.1cm]0:$p_6$}] at (p6) {};

    \coordinate (z) at (2,2);

    \draw [dashed,shorten >=-0.5cm] (z) -- (p0);
    \draw [dashed,shorten >=-0.5cm] (z) -- (p1);
    \draw [dashed,shorten >=-0.5cm] (z) -- (p2);
    \draw [dashed,shorten >=-0.5cm] (z) -- (p3);
    \draw [dashed,shorten >=-0.5cm] (z) -- (p4);
    \draw [dashed,shorten >=-0.5cm] (z) -- (p5);
    \draw [dashed,shorten >=-0.5cm] (z) -- (p6);

    \node [anchor=center,circle,draw,fill,inner
    sep=0.5pt,label={-170:$q$}] at (z) {};
  \end{tikzpicture}
  \caption{}
\end{figure}

Wyszukiwanie położenia danego punktu $z$ składa się z dwóch
etapów. Najpierw określany jest klin, w którym leży $z$. Można
to zrobić w czasie $O(\log n)$ używając wyszukiwania binarnego. Punkt
$z$ będzie znajdował się pomiędzy promieniami $p_i$ i $p_{i+n}$
podzielonej płaszczyzny wtedy i tylko wtedy, gdy kąt $\angle p_{i}qz$
będzie lewoskrętny, a kąt $\angle p_{i+n}qz$ prawoskrętny. W
następnych krokach stopniowo zawężamy nasz obszar poszukiwań, do czasu
aż odnajdziemy właściwy klin. Następnie określamy, czy $z$ należy
do wnętrza lub zewnętrza znalezionego klinu. Jeśli kąt $\angle
p_{i}p_{i+1}z$ jest lewoskrętny, to $z$ jest wewnętrzny. Określenie
skrętności kąta jest operacją o czasie $O(1)$, więc asymptotyczna
złożoność tej części algorytmu jest rzędu $O(\log N)$.

Umieszczenie wielokąta w strukturze danych umożliwiającej
przeszukiwanie binarne wymaga rozważenia przypadku zobrazowanego na
rysunku \ref{fig:binstruct}. By punkt $z$ został zakwalifikowany jako
należący do klina $(p_{3},p_{0})$ musi leżeć na przecięciu
półpłaszczyzny $H_1$ określonej przez obszar na prawo od prostej
zdefiniowanej przez odcinek $p_{3}q$ oraz półpłaszczyzny $H_2$
określonej przez obszar na lewo od prostej zdefiniowanej przez odcinek
$p_{0}q$. W sytuacji gdy kąt klina jest kątem rozwartym punkt $z$ nie
zostanie wykryty. Z tego powodu przy wyszukiwaniu binarnym powinniśmy
zaczynać od klina, którego kąt wewnętrzny jest mniejszy.

\begin{figure}[htp]
  \centering
  \begin{tikzpicture}
    \convexB

    \node [anchor=center,circle,draw,fill,inner sep=0.5pt,
    label={90:$p_0$}] at (p0) {};

    \node [anchor=center,circle,draw,fill,inner
    sep=0.5pt,label={100:$p_1$}] at (p1) {};

    \node [anchor=center,circle,draw,fill,inner
    sep=0.5pt,label={180:$p_2$}] at (p2) {};

    \node [anchor=center,circle,draw,fill,inner
    sep=0.5pt,label={[label distance=0.2cm]-90:$p_3$}] at (p3) {};

    \node [anchor=center,circle,draw,fill,inner
    sep=0.5pt,label={-45:$p_4$}] at (p4) {};

    \node [anchor=center,circle,draw,fill,inner
    sep=0.5pt,label={[label distance=0.1cm]0:$p_5$}] at (p5) {};

    \coordinate (q) at (2,2);
    \coordinate (z) at (3.25,3);

    \draw [dashed,shorten >=-0.5cm] (q) -- (p0);
    \draw [dashed,shorten >=-0.5cm] (q) -- (p1);
    \draw [dashed,shorten >=-0.5cm] (q) -- (p2);
    \draw [dashed,shorten >=-0.5cm] (q) -- (p3);
    \draw [dashed,shorten >=-0.5cm] (q) -- (p4);
    \draw [dashed,shorten >=-0.5cm] (q) -- (p5);

    \draw [name path=p4--q,green,dashed,shorten >=-4cm,shorten <=-2cm,->] (p3) -- (q);
    \draw [name path=p1--q,blue,dashed,shorten >=-4cm,shorten <=-1.5cm,->] (p0) -- (q);

    % \path [name intersections={of=p4--q and p1--q,by=i1}];

    % \fill [red] ($(q)!(-2,0)!(p1)$) circle [radius=2pt];
    % \fill [blue] ($(q)!(11,0)!(p1)$) circle [radius=2pt];

    \filldraw [color=blue,opacity=0.2] (6,5) --
    ($(q)!(11,0)!(p0)$) -- ($(q)!(-2,0)!(p0)$) -- (6,-1.2);

    % \fill [red] ($(p4)!(-2.5,0)!(q)$) circle [radius=2pt];
    % \fill [red] ($(p4)!(6.5,0)!(q)$) circle [radius=2pt];

    \filldraw [color=green,opacity=0.2] ($(p3)!(-2.5,0)!(q)$) --
    ($(p3)!(6.5,0)!(q)$) -- (6,-1.2) -- (-2.7,-1.2);

    \node [anchor=center,circle,draw,fill,inner
    sep=0.5pt,label={-170:$q$}] at (q) {};
    \node [anchor=center,circle,draw,fill,inner
    sep=0.5pt,label={below:$z$}] at (z) {};
  \end{tikzpicture}
  \caption{\label{fig:binstruct}}
\end{figure}

Zastosowane tutaj przeszukiwanie binarne jest możliwe dzięki temu, że
wierzchołki wielokąta wypukłego występują w kolejności kątowej lub
inaczej mówiąc w kolejności krążenia wokół punktu $q$. Przetwarzanie
wstępne dla wielokąta $P$, na które składa się wyznaczenie punktu $q$
oraz umieszczenie wielokąta w strukturze danych wspierającej
wyszukiwanie binarne, może zostać wykonane w czasie $O(n)$.

% todo: indeksowanie wierzchołków

%%% Local Variables:
%%% mode: latex
%%% TeX-master: "masterthesis"
%%% TeX-engine: xetex
%%% End:

\chapter{Wyznaczanie średnicy}
Jednym z zagadnień, w którym kluczowym jest wyznaczanie średnicy
zbioru punktów jest problem klastrowania. Klastrowanie zbioru to
grupowanie jego elementów o podobnych właściwościach. Można zauważyć,
że elementy klastra o mniejszej średnicy są ze sobą ściślej powiązane
niż w klastrze dużym. Przez wielkość klastra mamy na myśli jego
\emph{rozrzucenie}, czyli największą odległość między dwoma punktami
klastra. Jeśli mamy więc zbiór punktów podzielić na $K$ klastrów,
pożądany jest taki podział aby największa średnica klastra była
możliwie mała.

%%% Local Variables:
%%% mode: latex
%%% TeX-master: "masterthesis"
%%% TeX-engine: xetex
%%% End:

\chapter{Przecięcie wielokątów}
Problem znalezienia przecięcia dla wielokątów wypukłych zdefiniowany
jest następująco.

\begin{problem}[Przecięcie wielokątów]
  Mając dane dwa wielokąty wypukłe $P$ i $Q$, znaleźć ich
  \emph{przecięcie} $P \cap Q$, tj.\ obszar należący jednocześnie do
  $P$ i do $Q$.
\end{problem}

W niniejszym rozdziale do wielokąta będziemy zaliczać jego wnętrze
wraz z brzegiem. Jako $n$ będziemy oznaczać liczbę wierzchołków $P$, a
jako $m$ liczbę wierzchołków $Q$. Stosując algorytm naiwny polegający
na sprawdzeniu przecięcia każdej krawędzi wielokąta $P$ z każdą
krawędzią wielokąta $Q$ można uzyskać punkty wszystkich przecięć w
czasie $O(nm)$. Dla wielokątów prostych ten problem można rozwiązać w
czasie liniowo-logarytmicznym, uprzednio przekształcając go do
problemu przecięcia odcinków, które wymaga czasu
liniowego~\cite{Prep85}.

\section{Algorytm Shamosa-Hoeya}
Podstawową metodą rozwiązywania problemów w algorytmice jest technika
\emph{dziel i zwyciężaj}, które polega na podziale problemu na
mniejsze części, znajdowaniu rozwiązań częściowych, a następnie
łączeniu ich w końcowy rezultat.

Metodę wykorzystującą podział płaszczyzny dla problemu przecięcia
wielokątów zaproponowali Shamos i Hoey~\cite{ShamosHoey76}. Przez
każdy wierzchołek $P$ i $Q$ prowadzimy prostą pionową, tworząc w ten
sposób $n+m$ nieskończonych prostokątów zwanych dalej \emph{pasami}
(rysunek~\ref{img:ShamosHoey76}). Znalezienie przecięć krawędzi obydwu
wielokątów w każdej warstwie pozwoli nam na proste wyznaczenie obszaru
przecięcia tych wielokątów. Wynika to z poniższego twierdzenia.

\begin{twierdzenie}[Shamos-Hoey 1976]
  Przecięcie wielokątów wypukłych $P = (p_0, \ldots, p_{n-1})$ i $Q =
  (q_0, \ldots, q_{m-1})$ jest wielokątem wypukłym o liczbie
  wierzchołków nie większej niż $n + m$.
\end{twierdzenie}

\begin{proof}
  Przecięcie $P$ i $Q$ jest przecięciem $n + m$ wewnętrznym
  półpłaszczyzn wyznaczonych przez krawędzie obydwu wielokątów.
\end{proof}

Zauważmy, że w każdym pasie proste go ograniczające wraz z krawędziami
wielokąta tworzą trójkąt lub czworobok
(rysunek~\ref{img:ShamosHoey76}). Natomiast w pojedynczym pasie
przecięcie dwóch czworoboków (trójkątów) możemy wyznaczyć w czasie
stałym.

\begin{figure}[htb]
  \centering
  \includegraphics[scale=0.5]{img/ShamosHoey76}
  \caption{Podział na pasy a przecięcie
    wielokątów.\label{img:ShamosHoey76}}
\end{figure}

Następnie w czasie liniowym łączymy uzyskane kawałki i usuwamy
nadmiarowe wierzchołki powstałe przy podziale płaszczyzny. Początkowy
podział płaszczyzny również możemy przeprowadzić w czasie liniowym,
tak więc złożoność czasowa całego algorytmu wynosi $O(n + m)$.

\section{Algorytm O'Rourke-Chien-Olson-Naddor}
Kolejna metoda, zaproponowana przez O'Rourke i in.~\cite{Orourke98},
jest rozwinięciem algorytmu naiwnego.

Załóżmy, że $P$ i $Q$ przecinają się, oraz oznaczmy ich przecięcie $P
\cap Q$ jako $R$. Przyjmijmy również, że żadne dwie krawędzie z $P$ i
$Q$ nie są do siebie równoległe. Zauważmy, że brzeg $R$ składa się z
ciągu krawędzi należących do $P$, a następnie z ciągu krawędzie
należących do $Q$ (rysunek~\ref{img:sickles}). Jeżeli dany fragment
brzegu $R$ należy do $Q$, to jest on ,,objęty'' na zewnątrz przez ciąg
krawędzi $P$. I na odwrót --- fragmenty brzegu $R$, które należą do
$P$, są otoczone przez krawędzie należące do~$Q$.

\begin{figure}[htb]
  \centering
  \includegraphics[scale=0.5]{img/Orourke98}
  \caption{Sierpy a przecięcie wielokątów.\label{img:sickles}}
\end{figure}

Możemy zauważyć, że przechodząc w odpowiedniej kolejności,
jednocześnie po zewnętrznych i wewnętrznych krawędziach ,,sierpa''
(rysunek~\ref{img:sickles}), ograniczymy sprawdzanie przecięć jedynie
do krawędzi początkowych i końcowych danego sierpa. Z tego względu
przy wyborze krawędzi, do której mamy przejść w danym kroku, kierujemy
się tym, czy krawędź bieżąca może zawierać przecięcie, które dopiero
ma być wykryte. Algorytm można przedstawić w postaci pseudokodu z
listingu~\ref{alg:Orourke98}.

\begin{algorithm}
  \caption{Algorytm wyznaczania przecięcia metodą O'Rourke i
    in.\label{alg:Orourke98}}
  \begin{algorithmic}[1]
    \Procedure{Intersection of Convex Polygons}{}

    \State {$A \gets (p_0, p_1) \in P$}
    \State {$B \gets (q_0, q_1) \in Q$}

    \Repeat
    \State {sprawdź przecięcie krawędzi $A$ z krawędzią $B$}
    \State {w zależności od warunków przejdź po $A$ lub po $B$}
    \Until {$A$ i $B$ nie ,,okrążą'' swoich wielokątów}

    \EndProcedure
  \end{algorithmic}
\end{algorithm}

Wybierzmy dowolną krawędź $A \in P$ oraz dowolną krawędź $B \in
Q$. Będziemy przechodzić po krawędziach obu wielokątów w kierunku
przeciwnym do ruch wskazówek zegara. W każdym kroku sprawdzamy
przecięcie bieżących krawędzi wielokątów $P$ i $Q$, i wybieramy
krawędź, do której podążymy w następnym kroku.  Chcemy kierować się
następującą regułą: jeżeli krawędź $B$ jest ,,skierowana'' w kierunku
krawędzi $A$, ale jej nie przecina, to zbliżamy się do $A$ przechodząc
po krawędzi $B$ w miejsce możliwego przecięcia, w przeciwnym przypadku
przechodzimy po $A$.

\begin{figure}[htb]
  \centering
  \includegraphics[scale=0.7]{img/vectors}
  \caption{\label{img:advance} Wektory krawędzi wielokątów $P$ i $Q$.}
\end{figure}

Rozważmy sytuację z rysunku~\ref{img:advance}. Niech $H(A)$ oznacza
półpłaszczyznę po lewej stronie skierowanej prostej współliniowej do
krawędzi $A$. Pogrubiona strzałka to wektor krawędzi $A$, pozostałe to
wektory krawędzi $B$.  Jako $A \times B > 0$ zapiszmy warunek taki, że
współrzędna $z$ iloczynu wektorowego wektorów krawędzi $A$ i $B$ jest
dodatnia (na rysunku~\ref{img:advance} oznaczone linią ciągłą). Jako
$a$ oznaczmy ,,skierowany'' wierzchołek należący do krawędzi $A$, zaś
jako $b$ ,,skierowany'' wierzchołek należący do krawędzi $B$. Przy
wyborze krawędzi, po której przejdziemy w bieżącym kroku, stosujemy
poniższą regułę. Jeżeli

\begin{center}
  $A \times B > 0 $ i $b \notin H(A)$ lub $A \times B < 0 $ i $b \in
  H(A)$,
\end{center}

to przechodzimy po $B$. W przypadku przeciwnym przechodzimy po $A$.

Jeżeli $P$ i $Q$ przecinają się, to przecięcie zostanie znalezione po
$O(n+m)$ krokach. Znając punkty przecięć, dodatkowe $n+m$ kroków
wystarcza do wyznaczenia wszystkich wierzchołków tworzących brzeg
przecięcia $P \cap Q$. Możemy tym samym stwierdzić, że jeżeli po
$2(n+m)$ krokach nie odnajdziemy przecięcia, to krawędzie $P$ i $Q$
nie przecinają się. Na koniec, jeżeli nie odnotowano przecięcia
krawędzi, pozostaje sprawdzanie trzech końcowych warunków z
listingu~\ref{alg:OrourkeFinalTerms}. Sprawdzenie, czy punkt zawiera
się w wielokącie, tak jak to zostało pokazane w
rozdziale~\ref{chap:point_location}, dla wielokąta prostego można
wykonać w czasie liniowym, stąd złożoność czasowa całego algorytmu
wynosi $O(n + m)$.

\begin{algorithm}
  \caption{\label{alg:OrourkeFinalTerms} Procedura sprawdzająca, czy
    $P \subseteq Q$ lub $Q \subseteq P$, gdy nie wykryto przecięcia
    krawędzi.}
  \begin{algorithmic}[1]
    \If {$p_0$ leży w $Q$}
    \State $P$ zawiera się $Q$
    \Else
    \If {$q_0$ leży w $P$}
    \State $Q$ zawiera się $P$
    \Else
    \State $P$ i $Q$ nie przecinają się
    \EndIf
    \EndIf
  \end{algorithmic}
\end{algorithm}

\section{Algorytm Toussainta}
Stosunkowo prosty koncepcyjnie i implementacyjnie algorytm, działający
w czasie liniowym, zaproponował Toussaint w~\cite{ToussaintInt}.

\begin{figure}[htb]
  \centering
  \includegraphics[scale=0.7]{img/toussaint1}
  \caption{\label{img:toussaint1} Kieszenie a przecięcie wielokątów.}
\end{figure}

\subsection{Opis}
Rozważmy dwa przecinające się wypukłe wielokąty $P$ i $Q$ oraz otoczkę
wypukłą sumy tych wielokątów $CH(P \cup Q)$
(rysunek~\ref{img:toussaint1}). Niech brzegi $P$ i $Q$ przecinają się
w $k$ punktach $I_1, I_2, \ldots, I_k$ ponumerowanych zgodnie z ruchem
wskazówek zegara. Brzegi $P$ i $Q$ oraz otoczka wypukła $CH(P \cup Q)$
dzielą płaszczyznę na $2k + 1$ ograniczonych regionów: obszar
przecięcia wielokątów $P \cap Q$ wyznaczony przez punkty $I_1, I_2,
\ldots, I_k$, następnie $k$ regionów gdzie $P$ lub $Q$ należy do
otoczki wypukłej $CH(P \cup Q)$, ale nie należy do obszaru $P \cap Q$
przecięcia tych wielokątów, oraz analogiczne $k$ ,,kieszeni'' $K_1,
K_2, \ldots, K_k$ leżących wewnątrz otoczki wypukłej $CH(P \cup Q$),
ale leżących na zewnątrz $P \cup Q$. Każdej kieszeni $K_v$
przypisujemy odpowiadający jej punkt przecięcia $I_v$ oraz krawędź
otoczki $CH(P \cup Q)$, którą będziemy nazywać \emph{mostem}, łączącą
wierzchołek $p_i$ wielokąta $P$ oraz wierzchołek $q_j$ wielokąta $Q$,
i oznaczoną ją jako $B_{v}(p_{i_v}, q_{j_v})$. Cały algorytm
wyznaczający przecięcie wielokątów $P$ i $Q$ można przedstawić w
trzech krokach (listing~\ref{alg:interconpol}).

\begin{algorithm}
  \begin{algorithmic}[1]
    \Procedure{Alogrytm Interconpol}{}

    \State \emph{/* krok 1 */}

    \State wyznacz otoczkę wypukłą $CH(P \cup Q)$

    \State

    \If {$CH(P \cup Q) = P$ (lub $Q$)}
    \State zwróć $Q$ (lub $P$) jako przecięcie

    \Else
    \State kontynuuj
    \EndIf

    \State

    \State \emph{/* krok 2 */}

    \State \textbf{dla każdego} mostu z $CH(P \cup Q)$:
    \State \hspace{\algorithmicindent} znajdź odpowiadający mu punkt
    przecięcia
    \State \textbf{end}

    \State

    \State \emph{/* krok 3 */}

    \State połącz wewnętrzne łańcuchy należące do $P$ i $Q$

    wyznaczone przez punkty przecięcia znalezione w poprzednim kroku

    \EndProcedure
  \end{algorithmic}
  \caption{\label{alg:interconpol} Algorytm wyznaczający przecięcie
    wielokątów metodą Toussainta.}
\end{algorithm}


\subsection{Poprawność}
Poprawność algorytmu opiera się na poniższych lematach.

\begin{lemat}
  \em{\cite{ToussaintInt}} \em Jeżeli $P$ i $Q$ przecinają się to, dla
  każdego mostu $CH(P \cup Q)$ istnieje jeden związany z nim punkt
  przecięcia $P \cap Q$.
\end{lemat}

\begin{proof}
  Oznaczmy jako $L(u, v)$ skierowaną prostą przechodzącą przez $u$ i
  $v$ w kierunku z $u$ do $v$, oraz oznaczmy jako $RH(u, v)$
  półpłaszczyznę na prawo od $L(u, v)$. Analogicznie, jako $LH(u, v)$
  oznaczmy półpłaszczyznę leżącą na lewo od $L(u, v)$.

  Niech $B(p_i, q_j)$ będzie mostem
  (rysunek~\ref{img:toussaint2}). $L(p_i, q_j)$ jest prostą
  wspierającą dla $P$ i $Q$ jednocześnie, więc $P$ i $Q$ muszą leżeć w
  półpłaszczyźnie $RH(p_i, q_j)$. Przejdźmy po brzegu $P$ zaczynając
  od $p_i$ w kierunku zgodnym z ruchem wskazówek zegara, dopóki nie
  natrafimy na krawędź wielokąta $P$ przecinającą krawędź wielokąta
  $Q$. Punkt przecięcia oznaczmy jako $I$. Analogicznie przejdźmy po
  brzegu $Q$ zaczynając od $q_j$ w kierunku przeciwnym do ruchu
  wskazówek zegara, dopóki nie natrafimy na krawędź wielokąta $Q$
  przecinającą krawędź wielokąta $P$. Z wypukłości obu wielokątów
  wynika, że tym razem punktem przecięcia również jest $I$ i w ten
  sposób jest on związany z mostem $B(p_i, q_j)$.

  Z drugiej strony załóżmy, że $I$ jest pewnym punktem przecięcia
  pomiędzy krawędziami $(p_k, p_{k+1}) \in P$ i $(q_l, q_{l+1}) \in
  Q$. Ponieważ $P \in RH(p_k, p_{k+1})$ oraz $Q \in RH(q_l, q_{l+1})$,
  to żadna krawędź inna niż $(p_k, p_{k+1})$ i $(q_l, q_{l+1})$ nie
  może przecinać się w regionie $R = RH(p_{k+1}, p_k) \cap RH(q_{l+1},
  q_l)$. Dodatkowo, ponieważ $\angle p_{k}Iq_{l+1} < 180^{\circ}$,
  musi istnieć krawędź $(p_i, q_j) \in CH(P \cup Q)$, która przecina
  region $R$, i jest nią most odpowiadający punktowi przecięcia $I$.
\end{proof}

\begin{figure}[htb]
  \centering
  \includegraphics[scale=0.7]{img/toussaint2}
  \caption{\label{img:toussaint2} Most a punkt przecięcia.}
\end{figure}

\begin{twierdzenie}[Toussaint 1985\label{thm:toussaint85}]
  Opisany wyżej algorytm poprawnie wyznacza punkt przecięcia dla
  odpowiadającego mu mostu.
\end{twierdzenie}

Zanim omówimy dowód powyższego twierdzenia, wprowadźmy kilka
definicji.

\begin{definicja}
  \emph{Łańcuchem} $C(p_i,p_{i+1},\ldots,p_j)$ nazywamy ciąg
  kolejnych, następujących po sobie krawędzi i wierzchołków wielokąta
  prostego zaczynając od wierzchołka $p_i$ do wierzchołka
  $p_j$. Jeżeli każda krawędź łańcucha wraz z następną krawędzią
  tworzy kąt prawoskrętny, to taki łańcuch nazywamy
  \emph{wypukłym}. Analogicznie, jeżeli każda krawędź wraz ze swoim
  następnikiem tworzy kąt lewoskrętny, to taki łańcuch nazywamy
  \emph{łańcuchem wklęsłym}.
\end{definicja}

\begin{definicja}
  Wielokątem \emph{żaglokształtnym} nazywamy wielokąt zawierający
  krawędź $(p_{i},p_{i+1})$ będącą \emph{masztem} oraz wierzchołek
  $p_j$ (tzw.\ \emph{czubek żagla}) połączony z wierzchołkami $p_i$ i
  $p_{i+1}$ łańcuchami wklęsłymi.
\end{definicja}

\begin{definicja}
  Odcinek leżący w wielokącie $P$ i łączący dwa nie sąsiadujące
  wierzchołki wielokąta $P$ nazywamy \emph{przekątną} $P$.
\end{definicja}

\begin{definicja}\label{def:ear}
  Niech $p_i, p_{i+1}, p_{i+2}$ będą trzema kolejnymi, następującymi
  po sobie wierzchołkami należącymi do wielokąta $P$. Jeżeli przekątna
  łącząca $p_{i}$ i $p_{i+2}$ leży w $P$, to punkt $p_i$ nazywamy
  \emph{uchem} wielokąta $P$.
\end{definicja}

Zauważmy, że każda kieszeń wraz z odpowiadającym jej mostem oraz
punktem przecięcia $I_i$ tworzy wielokąt
\emph{żaglokształtny}. Ponadto mamy następujący lemat.

\begin{lemat}\label{lem:sailtip}\emph{\cite{ToussaintInt}}
  Czubek wielokąta żaglokształtnego jest uchem.
\end{lemat}

\begin{figure}[htb]
  \centering
  \includegraphics[scale=0.7]{img/toussaint3}
  \caption{\label{img:toussaint3} Wielokąt żaglokształtny.}
\end{figure}

\begin{proof}
  Przedłużmy krawędzie $(p_j,p_{j-1})$ i $(p_j, p_{j+1})$ tak, by
  przecinały prostą $L(p_i, p_{i+1})$ odpowiednio w punktach $x$ i $y$
  (rysunek~\ref{img:toussaint3}). Ze względu na to, że $p_j$ jest
  połączony z $p_i$ łańcuchem wypukłym, punkt $x$ musi leżeć na
  odcinku $(p_i, p_{i+1})$ będącym masztem $P$. Analogicznie, punkt
  $y$ musi leżeć na maszcie $P$. Punkty $p_j, p_{j-1}, x, y, p_{j+1},
  p_j$ tworzą obszar znajdujący się w całości w $P$, stąd przekątna
  $(p_{j-1}, p_{j+1})$ również musi leżeć w $P$.
\end{proof}

\begin{twierdzenie}[Meisters 1975\label{thm:meisters}] Każdy wielokąt
  o $n$ krawędziach $(n > 3)$ ma co najmniej dwoje nienachodzących na
  siebie uszu.
\end{twierdzenie}

\begin{lemat}\label{lem:sailmast}\emph{\cite{ToussaintInt}}
  Uchem jest góra masztu wielokąta żaglokształtnego albo uchem jest
  jego dół.
\end{lemat}

\begin{proof}
  Z definicji~\ref{def:ear} wiemy, że tylko wierzchołki wypukłe mogą
  być uszami, a więc w wielokącie żaglokształtnym $P$ uszami mogą być
  jedynie $p_i$, $p_{i+1}$ oraz $p_j$. Z lematu~\ref{lem:sailtip}
  wiemy, że $p_j$ musi być uchem. Z twierdzenia~\ref{thm:meisters}
  wiemy, że $P$ musi mieć co najmniej dwoje uszu. Stąd albo $p_i$ albo
  $p_{i+1}$ musi być uchem.
\end{proof}

\begin{proof}[Dowód twierdzenia~\ref{thm:toussaint85}]
  W każdym kroku algorytmu znajdującego przecięcie krawędzi wielokąta
  $P$ i wielokąta $Q$ zachowywany jest następujący niezmiennik: po
  obcięciu kolejnego ucha wielokąta żaglokształtnego pozostała cześć
  wielokąta również jest wielokątem żaglokształtnym.

  Lemat~\ref{lem:sailtip} pozwala nam na triangulację $P$ w każdym
  kroku obcinając ucho będące górą lub dołem masztu, dopóki nie
  dojdziemy do czubka żagla i krawędzi z niego wychodzących.
\end{proof}

W naszym przypadku wierzchołek $p_j$ jest szukanym punktem przecięcia
$I_i$, tak więc dodatkowo w algorytmie powinniśmy uwzględnić warunek
sprawdzający, czy w bierzącym kroku nadal rozważamy krawędź należącą
do wielokąta żaglokształtnego. Kompletny algorytm znajdujący czubek
żagla wielokąta $P$ został przedstawiony na
listingu~\ref{alg:stepdown}.

\begin{algorithm}
  \caption{\label{alg:stepdown} Algorytm znajdujący punkt przecięcia
    krawędzi wielokątów $P$ i $Q$ związany z danym mostem.}
  \begin{algorithmic}[1]
    \Procedure{Stepdown}{}

    \State $i \gets 1$
    \State $j \gets 1$

    % \State

    \Repeat
    \State $koniec \gets true$

    % \State

    \While{$\angle p_{i}p_{i+1}q_{j+1}$ jest lewoskrętny}
    \State $j \gets j + 1$
    \State $koniec \gets false$
    \EndWhile

    % \State

    \While{$\angle q_{j}q_{j+1}p_{i+1}$ jest prawoskrętny}
    \State $i \gets i + 1$
    \State $koniec \gets false$
    \EndWhile

    \Until{$koniec$}

    % \State

    \State $p_s \gets p_i$
    \State $q_t \gets q_j$

    \EndProcedure
  \end{algorithmic}
\end{algorithm}

Wejściem dla procedury jest most $B_k(p_i,p_j)$ należący do $CH(P \cup
Q)$, a wyjściem para wierzchołków $p_s$ i $q_t$, gdzie $(p_s,p_{s+1})
\cap (q_t,q_{t+1})$ wyznacza punkt przecięcia $I_k$. Warunki ,,$\angle
p_{i}p_{i+1}q_{j+1}$ jest lewoskrętny'' oraz ,,$\angle
q_{j}q_{j+1}p_{i+1}$ jest prawoskrętny'' w pętlach \textbf{while}
zapewniają, że w każdym kroku odetniemy ucho oraz że nie przejdziemy
przez czubek żagla. Przebieg działania procedury \textsc{Stepdown} dla
mostu z rysunku~\ref{fig:bridge} przedstawiają
rysunki~\ref{fig:stepdown1}-\ref{fig:stepdown5}.

\subsection{Złożoność}
Niech $n$ oznaczana liczbę wierzchołków wielokąta $P$, a $m$ liczbę
wierzchołków wielokąta $Q$. Znalezienie otoczki wypukłej dwóch
przecinających się wielokątów wypukłych może być wykonane w czasie
$O(m + n)$~\cite{Toussaint83}. W~\cite{ToussaintInt} Toussaint
proponuje wykorzystać technikę \emph{rotating calipers}, pozwalającą w
tym samym czasie stwierdzić, czy $CH(P \cup Q) = P$ lub $CH(P \cup Q)
= Q$.

Procedura znajdująca przecięcie $I_k$ jest wywoływana $k$ razy. Każde
wywołanie wymaga czasu liniowego zależnego od liczby wierzchołków
należących do danego żagla. Wszystkich takich wierzchołków może być
maksymalnie $n + m$, stąd ta część algorytmu wymaga czasu $O(n + m)$.

Ostatnią część algorytmu, tj.\ połączenie uzyskanych łańcuchów
wewnętrznych, można wykonać w czasie liniowym przechodząc przez
krawędzie obydwu wielokątów, jeżeli podczas szukania punktów przecięć
oznaczyliśmy kierunek, w którym dana część wielokąta staje się
łańcuchem wewnętrznym. Stąd złożoność czasowa całego algorytmu wynosi
$O(n + m)$.

\begin{figure}[tbp]
  \centering
  \begin{tikzpicture}
    \coordinate (q4) at (6, 5);
    \coordinate (q3) at (4, 4.5);
    \coordinate (q2) at (2.5, 3.5);
    \coordinate (q1) at (1.5, 1.5);
    \coordinate (q0) at (2, -0.25);
    \coordinate (q6) at (5, 0);
    \coordinate (q5) at (7, 2);

    \draw [blue] (q0) -- (q1) -- (q2) -- (q3) -- (q4) --  (q5) -- (q6) -- cycle;

    \node [anchor=center,circle,draw,fill,inner sep=1pt,
    label={250:$q_0$}] at (q0) {};

    \node [anchor=center,circle,draw,fill,inner sep=1pt,
    label={left:$q_1$}] at (q1) {};

    \node [anchor=center,circle,draw,fill,inner
    sep=1pt,label={right:$q_2$}] at (q2) {};

    \node [anchor=center,circle,draw,fill,inner
    sep=1pt,label={below:$q_3$}] at (q3) {};

    \node [anchor=center,circle,draw,fill,inner
    sep=1pt,label={above:$q_4$}] at (q4) {};

    \node [anchor=center,circle,draw,fill,inner
    sep=1pt,label={right:$q_5$}] at (q5) {};

    \node [anchor=center,circle,draw,fill,inner
    sep=1pt,label={below:$q_6$}] at (q6) {};

    \coordinate (p4) at (2, 3);
    \coordinate (p3) at (1, 4.5);
    \coordinate (p2) at (-1, 5);
    \coordinate (p1) at (-3, 3);
    \coordinate (p0) at (0, 0);
    \coordinate (p5) at (2, 1);

    \draw [red] (p0) -- (p1) -- (p2) -- (p3) --  (p4) -- (p5) -- cycle;

    \node [anchor=center,circle,draw,fill,inner sep=1pt,
    label={right:$p_1$}] at (p1) {};

    \node [anchor=center,circle,draw,fill,inner
    sep=1pt,label={above:$p_2$}] at (p2) {};

    \node [anchor=center,circle,draw,fill,inner
    sep=1pt,label={below:$p_3$}] at (p3) {};

    \node [anchor=center,circle,draw,fill,inner
    sep=1pt,label={left:$p_4$}] at (p4) {};

    \node [anchor=center,circle,draw,fill,inner
    sep=1pt,label={right:$p_5$}] at (p5) {};

    \node [anchor=center,circle,draw,fill,inner
    sep=1pt,label={below:$p_0$}] at (p0) {};

    \draw [dashed, shorten >= -2cm, shorten <= -2cm] (p2) -- (q4);
  \end{tikzpicture}
  \caption{Przerywaną linią oznaczono most.\label{fig:bridge}}
\end{figure}


\begin{figure}[tp]
  \centering
  \begin{tikzpicture}
    \coordinate (q4) at (6, 5);
    \coordinate (q3) at (4, 4.5);
    \coordinate (q2) at (2.5, 3.5);
    \coordinate (q1) at (1.5, 1.5);
    \coordinate (q0) at (2, -0.25);
    \coordinate (q6) at (5, 0);
    \coordinate (q5) at (7, 2);

    \draw [blue] (q1) -- (q2) -- (q3) -- (q4);

    \node [anchor=center,circle,draw,fill,inner sep=1pt,
    label={}] at (q1) {};

    \node [anchor=center,circle,draw,fill,inner
    sep=1pt,label={}] at (q2) {};

    \node [anchor=center,circle,draw,fill,inner
    sep=1pt,label={below:$q_{j+1}$}] at (q3) {};

    \node [anchor=center,circle,draw,fill,inner
    sep=1pt,label={above:$q_j$}] at (q4) {};

    \coordinate (p4) at (2, 3);
    \coordinate (p3) at (1, 4.5);
    \coordinate (p2) at (-1, 5);
    \coordinate (p1) at (-3, 3);
    \coordinate (p0) at (0, 0);
    \coordinate (p5) at (2, 1);

    \draw [red] (p2) -- (p3) --  (p4) -- (p5);

    \node [anchor=center,circle,draw,fill,inner
    sep=1pt,label={above:$p_i$}] at (p2) {};

    \node [anchor=center,circle,draw,fill,inner
    sep=1pt,label={left:$p_{i+1}$}] at (p3) {};

    \node [anchor=center,circle,draw,fill,inner
    sep=1pt,label={}] at (p4) {};

    \node [anchor=center,circle,draw,fill,inner
    sep=1pt,label={}] at (p5) {};

    \draw (p2) -- (q4);
    \draw [dashed] (p2) -- (q3);
  \end{tikzpicture}
  \caption{Kąt $\angle{p_{i}p_{i+1}q_{j+1}}$ jest lewoskrętny, więc
    odcinamy ucho $q_j$.\label{fig:stepdown1}}
\end{figure}

\begin{figure}[htp]
  \centering
  \begin{tikzpicture}
    \coordinate (q4) at (6, 5);
    \coordinate (q3) at (4, 4.5);
    \coordinate (q2) at (2.5, 3.5);
    \coordinate (q1) at (1.5, 1.5);
    \coordinate (q0) at (2, -0.25);
    \coordinate (q6) at (5, 0);
    \coordinate (q5) at (7, 2);

    \draw [blue] (q1) -- (q2) -- (q3) -- (q4);

    \node [anchor=center,circle,draw,fill,inner sep=1pt,
    label={}] at (q1) {};

    \node [anchor=center,circle,draw,fill,inner
    sep=1pt,label={}] at (q2) {};

    \node [anchor=center,circle,draw,fill,inner
    sep=1pt,label={below:$q_{j}$}] at (q3) {};

    \node [anchor=center,circle,draw,fill,inner
    sep=1pt,label={right:$q_{j+1}$}] at (q2) {};

    \coordinate (p4) at (2, 3);
    \coordinate (p3) at (1, 4.5);
    \coordinate (p2) at (-1, 5);
    \coordinate (p1) at (-3, 3);
    \coordinate (p0) at (0, 0);
    \coordinate (p5) at (2, 1);

    \draw [red] (p2) -- (p3) --  (p4) -- (p5);

    \node [anchor=center,circle,draw,fill,inner
    sep=1pt,label={above:$p_i$}] at (p2) {};

    \node [anchor=center,circle,draw,fill,inner
    sep=1pt,label={left:$p_{i+1}$}] at (p3) {};

    \node [anchor=center,circle,draw,fill,inner
    sep=1pt,label={}] at (p4) {};

    \node [anchor=center,circle,draw,fill,inner
    sep=1pt,label={}] at (p5) {};

    \draw (p2) -- (q3);
    \draw [dashed] (q3) -- (p3);
  \end{tikzpicture}
  \caption{Kąt $\angle{q_{j}q_{j+1},p_{i+1}}$ jest prawoskrętny, więc
    odcinamy ucho $p_i$.}
\end{figure}

\begin{figure}[htp]
  \centering
  \begin{tikzpicture}
    \coordinate (q4) at (6, 5);
    \coordinate (q3) at (4, 4.5);
    \coordinate (q2) at (2.5, 3.5);
    \coordinate (q1) at (1.5, 1.5);
    \coordinate (q0) at (2, -0.25);
    \coordinate (q6) at (5, 0);
    \coordinate (q5) at (7, 2);

    \draw [blue] (q1) -- (q2) -- (q3) -- (q4);

    \node [anchor=center,circle,draw,fill,inner sep=1pt,
    label={}] at (q1) {};

    \node [anchor=center,circle,draw,fill,inner
    sep=1pt,label={}] at (q2) {};

    \node [anchor=center,circle,draw,fill,inner
    sep=1pt,label={below:$q_{j}$}] at (q3) {};

    \node [anchor=center,circle,draw,fill,inner
    sep=1pt,label={right:$q_{j+1}$}] at (q2) {};

    \coordinate (p4) at (2, 3);
    \coordinate (p3) at (1, 4.5);
    \coordinate (p2) at (-1, 5);
    \coordinate (p1) at (-3, 3);
    \coordinate (p0) at (0, 0);
    \coordinate (p5) at (2, 1);

    \draw [red] (p2) -- (p3) --  (p4) -- (p5);

    \node [anchor=center,circle,draw,fill,inner
    sep=1pt,label={}] at (p2) {};

    \node [anchor=center,circle,draw,fill,inner
    sep=1pt,label={left:$p_{i}$}] at (p3) {};

    \node [anchor=center,circle,draw,fill,inner
    sep=1pt,label={left:$p_{i+1}$}] at (p4) {};

    \node [anchor=center,circle,draw,fill,inner
    sep=1pt,label={}] at (p5) {};

    \draw (q3) -- (p3);
    \draw [dashed] (p3) -- (q2);

  \end{tikzpicture}
  \caption{Kąt $\angle{p_{i}p_{i+1}q_{j+1}}$ jest lewoskrętny, więc
    odcinamy ucho $q_j$.}
\end{figure}

\begin{figure}[htp]
  \centering
  \begin{tikzpicture}
    \coordinate (q4) at (6, 5);
    \coordinate (q3) at (4, 4.5);
    \coordinate (q2) at (2.5, 3.5);
    \coordinate (q1) at (1.5, 1.5);
    \coordinate (q0) at (2, -0.25);
    \coordinate (q6) at (5, 0);
    \coordinate (q5) at (7, 2);

    \draw [blue] (q1) -- (q2) -- (q3) -- (q4);

    \node [anchor=center,circle,draw,fill,inner sep=1pt,
    label={250:$q_{j+1}$}] at (q1) {};

    \node [anchor=center,circle,draw,fill,inner
    sep=1pt,label={}] at (q3) {};

    \node [anchor=center,circle,draw,fill,inner
    sep=1pt,label={right:$q_{j}$}] at (q2) {};

    \coordinate (p4) at (2, 3);
    \coordinate (p3) at (1, 4.5);
    \coordinate (p2) at (-1, 5);
    \coordinate (p1) at (-3, 3);
    \coordinate (p0) at (0, 0);
    \coordinate (p5) at (2, 1);

    \draw [red] (p2) -- (p3) --  (p4) -- (p5);

    \node [anchor=center,circle,draw,fill,inner
    sep=1pt,label={}] at (p2) {};

    \node [anchor=center,circle,draw,fill,inner
    sep=1pt,label={left:$p_{i}$}] at (p3) {};

    \node [anchor=center,circle,draw,fill,inner
    sep=1pt,label={left:$p_{i+1}$}] at (p4) {};

    \node [anchor=center,circle,draw,fill,inner
    sep=1pt,label={}] at (p5) {};

    \draw (p3) -- (q2);
    \draw [dashed] (p4) -- (q2);
  \end{tikzpicture}
  \caption{Kąt $\angle{q_{j}q_{j+1}p_{i+1}}$ jest prawoskrętny, więc
    odcinamy ucho $p_i$.}
\end{figure}

\begin{figure}[htp]
  \centering
  \begin{tikzpicture}
    \coordinate (q4) at (6, 5);
    \coordinate (q3) at (4, 4.5);
    \coordinate (q2) at (2.5, 3.5);
    \coordinate (q1) at (1.5, 1.5);
    \coordinate (q0) at (2, -0.25);
    \coordinate (q6) at (5, 0);
    \coordinate (q5) at (7, 2);

    \draw [blue] (q1) -- (q2) -- (q3) -- (q4);

    \node [anchor=center,circle,draw,fill,inner sep=1pt,
    label={250:$q_{j+1}$}] at (q1) {};

    \node [anchor=center,circle,draw,fill,inner
    sep=1pt,label={}] at (q3) {};

    \node [anchor=center,circle,draw,fill,inner
    sep=1pt,label={right:$q_{j}$}] at (q2) {};

    \coordinate (p4) at (2, 3);
    \coordinate (p3) at (1, 4.5);
    \coordinate (p2) at (-1, 5);
    \coordinate (p1) at (-3, 3);
    \coordinate (p0) at (0, 0);
    \coordinate (p5) at (2, 1);

    \draw [red] (p2) -- (p3) --  (p4) -- (p5);

    \node [anchor=center,circle,draw,fill,inner
    sep=1pt,label={}] at (p2) {};

    \node [anchor=center,circle,draw,fill,inner
    sep=1pt,label={}] at (p3) {};

    \node [anchor=center,circle,draw,fill,inner
    sep=1pt,label={left:$p_{i}$}] at (p4) {};

    \node [anchor=center,circle,draw,fill,inner
    sep=1pt,label={right:$p_{i+1}$}] at (p5) {};

    \draw (p4) -- (q2);
  \end{tikzpicture}
  \caption{Kąt $\angle{p_{i}p_{i+1}q_{j+1}}$ jest prawoskrętny, a kąt
    $\angle{q_{j}q_{j+1}p_{i+1}}$ lewoskrętny, zatem algorytm kończy
    działanie.\label{fig:stepdown5}}
\end{figure}

%%% Local Variables:
%%% mode: latex
%%% TeX-master: "masterthesis"
%%% TeX-engine: xetex
%%% End:

\chapter{Metoda \emph{rotating calipers}\label{chap:calipers}}
Użyty w rozdziale~\ref{chap:diameter} algorytm można zobrazować jako
zestaw obracających się suwmiarek i jako ogólną metodę można
zastosować przy rozwiązywaniu innych problemów geometrycznych dla
wielokątów wypukłych; pozwala to na uzyskanie algorytmów działających
w czasie liniowym.

\section{Wyznaczanie średnicy}
W tej sekcji zostanie przedstawiony algorytm wyznaczania średnicy z
poprzedniego rozdziału z użyciem techniki \emph{rotating calipers}.

Niech $P = (p_0, p_1, \ldots, p_{n-1})$ będzie wielokątem wypukłym
oraz niech jego wierzchołki będą ponumerowane zgodnie z ruchem
wskazówek zegara. Algorytm z rozdziału~\ref{chap:diameter} znajduje
wszystkie pary antypodyczne, a następnie wybiera z nich najbardziej od
siebie oddalona parę wierzchołków definiujących średnicę (na podstawie
twierdzenia~\ref{thm:yagbol}). Do znalezienia punktów antypodycznych
możemy użyć suwmiarki składającej się z dwóch równoległych prostych
wspierających. Rozważmy rysunek~\ref{img:calipers1}. Jako początkową
pozycję prostych wspierających wyznaczmy prostą równoległą do osi $x$
przechodzącą przez wierzchołek wielokąta $P$ położony najwyżej
względem osi $y$ oraz równoległą do niej prostą przechodzącą przez
wierzchołek wielokąta $P$ położony najniżej względem osi $y$. Punkty
$p_i$ i $p_j$ są pierwszą znalezioną parą antypodyczną. Aby znaleźć
następną, rozważmy kąty, które tworzy prosta wspierająca przechodząca
przez $p_i$ z krawędzią $(p_i, p_{i+1})$ oraz prosta wspierająca
przechodząca przez $p_j$ z krawędzią $(p_j, p_{j+1})$.  Jeżeli
$\angle{\theta_j} < \angle{\theta_i}$, to obracamy proste wspierające
o kąt $\theta_j$, w przeciwnym przypadku obracamy proste wspierające o
kąt $\theta_i$. Para $(p_{j+1}, p_i)$ staje się następną parą
antypodyczną. Proces powtarzamy do momentu, gdy proste wspierające
zamienią się miejscami, tj.\ gdy $p_j = p_{i_0}$ i $p_i = p_{j_0}$,
gdzie $p_{i_0}$ to wierzchołek styczny do początkowej pozycji
pierwszej prostej wspierającej, a $p_{j_0}$ to wierzchołek styczny do
początkowej pozycji drugiej prostej wspierającej.

\begin{figure}[tb]
  \centering
  \includegraphics[scale=0.5]{img/calipers1}
  \caption{\label{img:calipers1} Wierzchołki $p_i$ i $p_j$ tworzą parę
    antypodalną.}
\end{figure}

\section{Najmniejszy prostokąt zawierający}
Problem najmniejszego prostokąta zawierającego zdefiniowany jest
następująco.

\begin{problem}[Najmniejszy prostokąt zawierający]
  Dla wielokąta wypukłego $P$ znaleźć najmniejszy prostokąt taki, żeby
  $P$ był w nim zawarty.
\end{problem}

Rozwiązanie tego problemu ma swoje zastosowanie m.\ in.\ w
przetwarzaniu obrazów, grach komputerowych i niektórych algorytmach
dotyczących optymalnego upakowania.

Algorytm z wykorzystaniem suwmiarek opiera się na następującym
twierdzeniu.

\begin{twierdzenie}[Freeman-Shapira 1975]
  Najmniejszy prostokąt zawierający wielokąt $P$ posiada bok
  współliniowy z bokiem $P$.
\end{twierdzenie}

Jako $L_s(p_i)$ będziemy oznaczać skierowaną prostą wspierającą
wielokąta w wierzchołku $p_i$ taką, że $P$ znajduje się po prawej
stronie prostej. Rozważmy rysunek~\ref{img:calipers2}. W pierwszym
kroku znajdujemy wierzchołki o minimalnych i maksymalnych
współrzędnych $x$ i $y$. Oznaczmy te wierzchołki jako $p_i$, $p_j$,
$p_k$, $p_l$, w kolejności zgodnej z ruchem wskazówek zegara. Niech
para prostych $L_s(p_j)$ i $L_s(p_l)$ będzie pierwszą suwmiarką,
natomiast para $L_s(p_i)$ i $L_s(p_k)$ drugą. Analogicznie jak w
algorytmie wyznaczającym średnicę, mamy tym razem do rozważenia cztery
kąty: $\angle{\theta_i}, \angle{\theta_j}, \angle{\theta_k},
\angle{\theta_l}$. Obróćmy teraz wszystkie cztery proste wspierające o
najmniejszy z nich. Wyznaczmy pole prostokąta wyznaczonego przez
punkty przecięć prostych wspierających $L_s(p_i), L_s(p_j), L_s(p_k)$
i $L_s(p_l)$. Czynność tę powtarzamy z nowo powstałymi kątami
$\angle{\theta_i}, \angle{\theta_j}, \angle{\theta_k},
\angle{\theta_l}$ do czasu, gdy rozpatrzone zostaną wszystkie możliwe
prostokąty zawierające, tj.\ obydwie suwmiarki wykonają pełny obrót
wokół $P$. Ze wszystkich wyznaczonych w ten sposób prostokątów
wybieramy najmniejszy.

\begin{figure}[tb]
  \centering
  \includegraphics[scale=0.5]{img/calipers2}
  \caption{\label{img:calipers2} Pary wierzchołków $(p_i, p_k)$ i
    $(p_j, p_l)$ są parami antypodycznymi.}
\end{figure}

\section{Największa odległość\label{sec:max_dist}}
\begin{problem}
  Niech $P = (p_0, p_1, \ldots, p_{n-1})$ oraz $Q = (q_0, q_1, \ldots,
  q_{m-1})$ będą wielokątami wypukłymi. Największą odległość między
  $P$ i $Q$ oznaczamy jako $d_{\max}(P, Q)$ i definiujemy
  jako: $$d_{\max}(P, Q) := \max{\{ d(p_i, q_j) \mid i = 0, 1, \ldots,
    n - 1; j = 0, 1, \ldots, m - 1\}},$$ gdzie $d(p_i, q_j)$ oznacza
  odległość euklidesową pomiędzy $p_i$ i $q_j$.
\end{problem}

Należy zauważyć, że największa odległość między dwoma wielokątami
wypukłymi nie musi być równa średnicy otoczki wypukłej $P \cup Q$
(rysunek~\ref{fig:maxdist}), więc nie możemy wykorzystać algorytmu
wyznaczającego średnicę do rozwiązania tego problemu.

\begin{figure}[tb]
  \centering
  \begin{tikzpicture}
      \coordinate (p3) at (3,5);
      \coordinate (p2) at (1.5,3);
      \coordinate (p1) at (2,1);
      \coordinate (p0) at (4,-1);

      \draw  (p0) -- (p1) -- (p2) -- (p3);

      \node [anchor=center,circle,draw,fill,inner
      sep=1pt,label={below:$p_0$}] at (p0) {};

      \node [anchor=center,circle,draw,fill,inner
      sep=1pt,label={left:$p_1$}] at (p1) {};

      \node [anchor=center,circle,draw,fill,inner
      sep=1pt,label={left:$p_2$}] at (p2) {};

      \node [anchor=center,circle,draw,fill,inner
      sep=1pt,label={left:$p_3$}] at (p3) {};

      \coordinate (q0) at (5,0);
      \coordinate (q2) at (5,3);
      \coordinate (q1) at (4,1);

      \node [anchor=center,circle,draw,fill,inner
      sep=1pt,label={below:$q_0$}] at (q0) {};

      \node [anchor=center,circle,draw,fill,inner
      sep=1pt,label={below:$q_1$}] at (q1) {};

      \node [anchor=center,circle,draw,fill,inner
      sep=1pt,label={above:$q_2$}] at (q2) {};

      \draw (q0) -- (q1) -- (q2) -- cycle;

      \draw [blue] (p0) -- (p3);

      \draw [red] (p3) -- (q0);

      \draw [dashed] (p3) -- (q2);
      \draw [dashed] (p0) -- (q0);
  \end{tikzpicture}
  \caption{\label{fig:maxdist} Odcinek $(p_0, p_3)$ jest średnicą otoczki
    wypukłej $P \cup Q$, natomiast odcinek $(p_3, q_0)$ jest największą
    odległością między tymi wielokątami.}
\end{figure}

Rozważmy rysunek~\ref{img:calipers3}. W pierwszym kroku znajdujemy dwa
punkty, z których jeden $p_i$ należy do $P$, a drugi $q_j$ do $Q$, tak
aby proste wspierające $L_s(p_i)$ i $L_s(p_j)$ były równoległe. Możemy
na przykład wybrać punkty o skrajnym położeniu względem osi $y$. Niech
$L_s(p_i)$ i $L_s(q_j)$ będą prostymi skierowanymi w przeciwnych
kierunkach. Utworzone w ten sposób proste wspierające tworzą kąt
$\angle{\theta_i}$ z krawędzią $(p_i, p_{i+1}) \in P$ oraz kąt
$\angle{\phi_j}$ z krawędzią $(q_j, q_{j+1}) \in Q$. Wyznaczamy i
zapamiętujemy odległość między $p_i$ i $q_j$. Wybieramy mniejszy z
kątów $\phi_j$ i $\theta_i$, a następnie obracamy obydwie proste
wspierające o ten kąt. Proces powtarzamy do czasu, gdy proste
wspierające zamienią się początkowymi pozycjami. W ostatnim kroku
wybieramy największą z wyznaczonych odległości między punktami.

\begin{figure}[tb]
  \centering
  \includegraphics[scale=0.5]{img/calipers3}
  \caption{\label{img:calipers3} Para wierzchołków $(p_i, q_j)$ jest
    parą antypodalną między wielokątami $P$~i~$Q$.}
\end{figure}

\section{Łączenie otoczek wypukłych}
\begin{problem}
  Dla $CH(P)$ i $CH(Q)$ będącymi otoczkami wypukłymi zbiorów $P$ i
  $Q$ znaleźć otoczkę wypukłą $CH(CH(P) \cup CH(Q))$.
\end{problem}

W algorytmach typu \emph{dziel i zwyciężaj} wyznaczających otoczkę
wypukłą zbioru punktów końcowym etapem jest łączenie otoczek
uzyskanych w poprzednim kroku, uzyskując w ten sposób coraz większe
otoczki, aż do uzyskania otoczki wypukłej całego zbioru. Jeśli łącznie
zostałoby przeprowadzone w czasie liniowym, złożoność czasowa całego
algorytmu w najlepszym przypadku będzie
liniowo-logarytmiczna~\cite{Graham72}.

\begin{figure}[htb]
  \centering
  \includegraphics[scale=0.5]{img/calipers4}
  \caption{\label{img:calipers4} Wierzchołki $p_i$ i $q_j$ tworzą parę
    kopodalną.}
\end{figure}

Rozważmy dwa wielokąty wypukłe $P = (p_0, \ldots, p_{n-1})$ i $Q =
(q_0, \ldots, q_{n-1})$, gdzie wierzchołki $P$ i $Q$ są numerowane
zgodnie z kierunkiem ruchu wskazówek zegara, będącymi otoczkami
wypukłymi pewnych zbiorów punktów (rysunek~\ref{img:calipers4}). W tym
przypadku wyznaczenie otoczki wypukłej obydwu wielokątów wymaga
znalezienia dwóch par wierzchołków $(p_i, p_j)$, $(q_k, q_l)$ takich,
że krawędzie $(p_i, q_k)$ i $(p_j, q_l)$ wraz z łańcuchami wypukłymi
$(p_i, p_{i+1}, \ldots, p_j)$ i $(q_k, q_{k+1}, \ldots, q_l)$ tworzą
$CH(P \cup Q)$. Krawędzie $(p_i, q_k)$ i $(p_j, q_l)$ będziemy nazywać
\emph{mostami}, a wierzchołki tworzące most będziemy nazywać
\emph{punktami mostu}. Ponadto mówimy, że para punktów $p \in P$ i $q
\in Q$ jest \emph{parą kopodalną między $P$ i $Q$}, jeżeli można
skonstruować równoległe proste wspierające styczne do wierzchołków $p$
i $q$ odpowiednio.

Algorytm wykorzystujący suwmiarki do wyznaczania mostów opiera się na
następującym twierdzeniu~\cite{Toussaint83}.

\begin{twierdzenie}[Toussaint 1983]
\label{thm:bridge}
Dwa wierzchołki $p_i \in P$ i $q_j \in Q$ są punktami mostu wtedy i
tylko wtedy, gdy tworzą parę kopodalną i wierzchołki $p_{i-1},
p_{i+1}, q_{j-1}$, $ q_{j+1}$ leżą po tej samej stronie prostej
$L(p_i, q_j)$.
\end{twierdzenie}

Po wyznaczeniu początkowych pozycji suwmiarek, analogicznie jak w
algorytmie \ref{sec:max_dist}, skierowanych w przeciwnych kierunkach,
zaczynamy obracać je wokół wielokątów. Podobnie jak w poprzednich
problemach rozważamy kąt, jaki tworzy skierowana prosta wspierająca
wielokąta $P$ w punkcie $p_i$ z krawędzią $(p_{i+1}, p_{i+2})$ oraz
prosta wspierająca wielokąta $Q$ z krawędzią $(q_{j+1},
q_{j+2})$. Obydwie suwmiarki obracamy o mniejszy z tych kątów. Przed
każdym kolejnym obrotem sprawdzamy, czy aktualna para wierzchołków
$(p_i, q_j)$ spełnia warunek zawarty w
twierdzeniu~\ref{thm:bridge}. Ponieważ przy każdym obrocie suwmiarek
generowana jest nowa para kopodalna, pozostaje sprawdzić, czy
wierzchołki $p_{i-1}, p_{i+1}, q_{j-1}, q_{j+1}$ leżą po tej samej
stronie prostej $L(p_i, q_j)$. Algorytm znajdowania mostów w przypadku
łączenia otoczek wypukłych można zakończyć po znalezieniu obydwu
mostów.

Niech liczba wierzchołków $P$ wynosi $n$, a liczba wierzchołków $Q$
wynosi $m$. Par kopodalnych jest mniej niż $n + m$. Sprawdzenie, czy
para kopodalna jest mostem, odbywa się w czasie $O(1)$, stąd złożoność
czasowa algorytmu wyznaczającego mosty jest rzędu $O(n + m)$.

%%% Local Variables:
%%% mode: latex
%%% TeX-master: "masterthesis"
%%% TeX-engine: xetex
%%% End:

\chapter{Zawieranie wielokątów}
Ogólny problem zawierania wielokątów formułujemy następująco.

\begin{problem}
  Dla dwóch wielokątów prostych $P$ i $Q$ stwierdzić, czy $P$ może być
  zawarty w $Q$, i jeżeli tak, to podać \emph{umiejscowienie} $P$,
  które spełnia zawieranie.
\end{problem}

\begin{figure}[htb]
  \centering
  \begin{tikzpicture}
      \coordinate (p0) at (5.5,1.5);
      \coordinate (p1) at (3,4);
      \coordinate (p2) at (0,3);
      \coordinate (p3) at (-1,-0.5);
      \coordinate (p4) at (3,-2);

      \draw (p0) -- (p1) -- (p2) -- (p3) -- (p4) -- cycle;

      \node [anchor=center,circle,draw,fill,inner
      sep=1pt,label={right:$p_0$}] at (p0) {};

      \node [anchor=center,circle,draw,fill,inner
      sep=1pt,label={above:$p_1$}] at (p1) {};

      \node [anchor=center,circle,draw,fill,inner
      sep=1pt,label={left:$p_2$}] at (p2) {};

      \node [anchor=center,circle,draw,fill,inner
      sep=1pt,label={left:$p_3$}] at (p3) {};

      \node [anchor=center,circle,draw,fill,inner
      sep=1pt,label={below:$p_4$}] at (p4) {};

      \coordinate (q0) at (8,6);
      \coordinate (q1) at (4,7);
      \coordinate (q2) at (4,4);
      \coordinate (q3) at (6,4);

      \draw (q0) -- (q1) -- (q2) -- (q3) -- cycle;

      \node [anchor=center,circle,draw,fill,inner
      sep=1pt,label={right:$q_0$}] at (q0) {};

      \node [anchor=center,circle,draw,fill,inner
      sep=1pt,label={60:$q_1$}] at (q1) {};

      \node [anchor=center,circle,draw,fill,inner
      sep=1pt,label={left:$q_2$}] at (q2) {};

      \node [anchor=center,circle,draw,fill,inner
      sep=1pt,label={below:$q_3$}] at (q3) {};

      \coordinate (q0') at (4.5, 1.5);
      \coordinate (q1') at (0.5, 2.5);
      \coordinate (q2') at (0.5, -0.5);
      \coordinate (q3') at (2.5, -0.5);

      \draw [dashed] (q0') -- (q1') -- (q2') -- (q3') -- cycle;

      \node [anchor=center,circle,draw,fill,inner
      sep=1pt,label={right:$q_{0}'$}] at (q0') {};

      \node [anchor=center,circle,draw,fill,inner
      sep=1pt,label={60:$q_{1}'$}] at (q1') {};

      \node [anchor=center,circle,draw,fill,inner
      sep=1pt,label={left:$q_{2}'$}] at (q2') {};

      \node [anchor=center,circle,draw,fill,inner
      sep=1pt,label={below:$q_{3}'$}] at (q3') {};
  \end{tikzpicture}
  \caption{Wielokąt $(q_0, \ldots, q_3)$ można umieścić w wielokącie
    $(p_0, \ldots, p_4)$\label{img:containment1}}
\end{figure}

W problemie ogólnym dopuszczamy translacje i obroty zawieranego
wielokąta tak, by ,,zmieścił się'' on wielokącie
zawierającym. Rozwiązanie tego problemu w czasie $O(pq^2)$ w
przypadku, gdy $Q$ jest wielokątem wypukłym przedstawił Chazelle
w~\cite{Chazelle83}. Dla dowolnych wielokątów rozwiązanie naiwne
wymaga czasu $O[p^3q^3(p + q) \log{(p + q)}]$~\cite{Chazelle83}. W tej
samej pracy przedstawiono także rozwiązanie dla uproszczonego
problemu, w którym wielokąty $P$ i $Q$ są wypukłe, a dozwolonym
działaniem na $P$ jest jedynie translacja.

Niech będzie dany wielokąt wypukły $P = (p_0, p_2, \ldots, p_{n-1})$
oraz wielokąt wypukły $Q = (q_0, q_1, \ldots, q_{m-1})$. Dla
uproszczenia zakładamy, że żadna para krawędzi wielokątów $P$ i $Q$
nie jest do siebie równoległa. Niech $H_i$ dla $i = 0, \ldots, m - 1$
będzie $i$-tą półpłaszczyzną określoną przez prostą współliniową do
krawędzi $(q_i, q_{i+1})$ i zawierającą wielokąt $Q$. Niech punkt $c$
będzie dowolnym punktem należącym do $P$, może być to na przykład
środek jego masy. Punkt ten będzie wyznaczał umiejscowienie $P$ na
płaszczyźnie.

Gdy przesuniemy $P$ tak, by punkt $c$ znajdował się w jak najmniejszej
odległości od $H_i$, to jeden wierzchołek $p_j \in P$ będzie styczny z
prostą $L(q_i, q_{i+1})$ wyznaczającą półpłaszczyznę $H_i$. O takim
wierzchołku $p_j$ mówimy, że jest \emph{krytyczny} dla krawędzi $(q_i,
q_{i+1})$. Odległość punktu $c$ od półpłaszczyzny $H_i$ oznaczmy jako
$d_i$.

Rozważmy półpłaszczyznę $H_i'$ wyznaczoną przez prostą
$L'(q_i,q_{i+1})$ równoległą do $L(q_i, q_{i+1})$ i przechodzącą przez
punkt $c$. $L'(q_i,q_{i+1})$ możemy łatwo wyznaczyć, jeśli znamy
wierzchołek krytyczny dla $(q_i, q_{q+1})$. Zauważmy, że $P$ jest
zawarty w $Q$ wtedy i tylko wtedy, gdy jest zawarty w przecięciu
półpłaszczyzn $H_0 \cap \ldots \cap H_{m-1}$. Warunek ten jest
równoważny zawieraniu punktu $c$ w przecięciu półpłaszczyzn
$H_i'$. Innymi słowy: wielokąt $P$ zawiera się w $Q$ dokładnie wtedy,
gdy punkt $c$ należy do przecięcia półpłaszczyzn $H_i'$. Co więcej,
wspomniane przecięcie zawiera wszystkie możliwe umiejscowienia $c$
spełniające zawieranie $P$ w $Q$. Algorytm rozwiązujący problemu
zawierania wielokąta $P$ w $Q$ możemy przedstawić następująco.

\begin{figure}[htp]
\begin{algorithmic}[1]
\Procedure{Convex Polygon Containment}{}

\State \textbf{dla każdej} krawędzi $(q_i, q_{i+1}) \in Q$
\State \hspace{\algorithmicindent} wyznacz wierzchołek krytyczny $p_j \in P$
\State \hspace{\algorithmicindent} wyznacz półpłaszczyznę $H_i'$
\State \textbf{end}
\State wyznacz przecięcie otrzymanych półpłaszczyzn $H_i$

\EndProcedure
\end{algorithmic}
\end{figure}

Do znalezienia wierzchołków krytycznych możemy się posłużyć
przedstawioną w rozdziale \ref{chap:calipers} metodą \emph{rotating
  calipers}. Zauważmy, że dana krawędź $(q_i, q_{i+1})$ wraz z
odpowiadającym jej wierzchołkiem krytycznym $p_j$ tworzą parę
kopodalną. Niech początkowymi prostymi wspierającymi dla $P$ i $Q$
będą równoległe proste poziome, styczne do wierzchołka o najmniejszej
wartości współrzędnej $x$ z $P$ i $Q$ odpowiednio. Niech obydwie
proste wspierające będą skierowane w tym samym kierunku w ten sposób,
żeby wspierany wielokąt znajdował się po lewej stronie prostej
(rysunek~\ref{img:containment3}). Analogicznie jak w
rozdziale~\ref{chap:calipers} zaczynamy obracać równoległe proste
wspierające. Przy każdym obrocie suwmiarek, po którym prosta
wspierająca $L_{SQ}$ jest styczna z krawędzią $Q$, prosta wspierająca
$L_{SP}$ jest styczna z punktem krytycznym dla tej krawędzi. Następnie
dla każdej krawędzi $Q$ i odpowiadającemu jej punktowi krytycznemu
wyznaczamy półpłaszczyznę $H_i'$, tak jak zostało to opisane
wcześniej.

\begin{figure}[tb]
  \centering
  \begin{tikzpicture}
      \coordinate (p0) at (5.5,1.5);
      \coordinate (p1) at (3,4);
      \coordinate (p2) at (0,3);
      \coordinate (p3) at (-1,-0.5);
      \coordinate (p4) at (3,-2);

      \draw (p0) -- (p1) -- (p2) -- (p3) -- (p4) -- cycle;

      \node [anchor=center,circle,draw,fill,inner
      sep=1pt,label={right:$p_0$}] at (p0) {};

      \node [anchor=center,circle,draw,fill,inner
      sep=1pt,label={above:$p_1$}] at (p1) {};

      \node [anchor=center,circle,draw,fill,inner
      sep=1pt,label={left:$p_2$}] at (p2) {};

      \node [anchor=center,circle,draw,fill,inner
      sep=1pt,label={left:$p_3$}] at (p3) {};

      \node [anchor=center,circle,draw,fill,inner
      sep=1pt,label={below:$p_4$}] at (p4) {};

      \coordinate (q0) at (11,2);
      \coordinate (q1) at (7,3);
      \coordinate (q2) at (7,0);
      \coordinate (q3) at (9,0);

      \draw (q0) -- (q1) -- (q2) -- (q3) -- cycle;

      \node [anchor=center,circle,draw,fill,inner
      sep=1pt,label={right:$q_0$}] at (q0) {};

      \node [anchor=center,circle,draw,fill,inner
      sep=1pt,label={60:$q_1$}] at (q1) {};

      \node [anchor=center,circle,draw,fill,inner
      sep=1pt,label={left:$q_2$}] at (q2) {};

      \node [anchor=center,circle,draw,fill,inner
      sep=1pt,label={below:$q_3$}] at (q3) {};

      \draw [blue, <-, shorten <= -2cm, shorten >= -2cm] (p2) -- (p1);

      \draw[blue, ->, shorten <= -4cm] (q1) -- +($(p2)-(p1)$);
  \end{tikzpicture}
  \caption{Pary punktów $(p_1,q_1)$ oraz $(p_2, q_1)$ to pary
    kopodalne.}
\end{figure}

Na koniec pozostaje wyznaczenie przecięcia półpłaszczyzn $H_1' \cap
\ldots \cap H_m'$. Uzyskane półpłaszczyzny podzielmy dwa zbiory: dolny
i górny. Niech w dolnym zbiorze znajdą się półpłaszczyzny wyznaczone
przez ,,dolną'' część wielokąta $Q$. Zacznijmy rozpatrywać kolejne
półpłaszczyzny z dolnego zbioru pod względem nierosnącego
współczynnika kierunkowego prostej $l_i$ określającej półpłaszczyznę
$H_i'$. Włóżmy dwie pierwsze proste na stos. Będziemy oznaczać
ostatnią prostą na stosie jako $l_o$, a przedostatnią jako $l_p$. Po
rozpatrzeniu każdej kolejnej półpłaszczyzny będziemy zachowywać
następujący niezmiennik: na stosie znajdują się te proste z ciągu
$l_1, \ldots, l_i$, które stanowią brzeg zbioru $H_1' \cap \ldots \cap
H_i'$. Postępujemy zgodnie z regułą, że jeżeli przecięcie obecnie
rozważanej prostej $l_i$ z ostatnia prostą na stosie $l_o$ leży na
lewo od przecięcia $l_o$ z przedostatnią prostą $l_p$ ze stosu, to
zdejmujemy $l_o$ ze stosu i wkładamy na stos prostą $l_i$. Tę metodę
obrazuje sytuacja z rysunku~\ref{img:containment2}. Na stos zostały
włożone proste $l, l_p, l_o$, rozważana jest prosta $l_i$. Punkt $q' =
l_i \cap l_o$ leży na lewo od $q = l_o \cap l_p$. Wobec tego
zdejmujemy nadmiarową prostą $l_o$ ze stosu, a wkładamy na niego
prostą $l_i$. Analogicznie postępujemy z zbiorem ,,górnych''
półpłaszczyzn. W ostatnim kroku łączymy uzyskane dwa obszary wypukłe,
poprzez wyznaczenie przecięcia par skrajnych półpłaszczyzn z
,,dolnej'' i ,,górnej'' otoczki.

\vspace{4cm}

\begin{figure}[ht]
  \centering
  \includegraphics[scale=0.6]{img/containment2}
  \caption{\label{img:containment2} Czerwoną linią zaznaczono brzeg
    przecięcia półpłaszczyzn.}
\end{figure}

\cleardoublepage{}

\section{Poprawność}
Poprawność warunku zawierania $P$ w $Q$ jako warunku zawierania $P$ w
przecięciu półpłaszczyzn wyznaczonych przez krawędzie $Q$ opiera się
na następującym lemacie.

\begin{lemat}\emph{\cite{Chazelle83}}
  Przecięcie $n$ półpłaszczyzn jest obszarem wypukłym.
\end{lemat}

Lemat ten wynika bezpośrednio z poniższych własności.

\begin{itemize}
  \item Półpłaszczyzna jest zbiorem wypukłym.
  \item Przecięcie zbiorów wypukłych jest zbiorem wypukłym.
\end{itemize}

Drugim lematem, na którym opiera się algorytm jest:

\begin{lemat}\emph{\cite{Chazelle83}}
  Dla każdej krawędzi $Q$ istnieje wierzchołek krytyczny z $P$.
\end{lemat}

Zakładając, że żadne trzy wierzchołki $P$ nie są współliniowe, to z
wypukłości $P$ wynika, że istnieją co najwyżej dwa najbardziej
zbliżone do krawędzi $(q_i,q_{i+1})$ wierzchołki. Ma to miejsce w
przypadku, gdy krawędź $(q_i,q_{i+1})$ jest równoległa do krawędzi $P$
zawierającej obydwa wierzchołki krytyczne. W przeciwnym przypadku
istnieje dokładniej jeden wierzchołek krytyczny.

\begin{lemat}\emph{\cite{Chazelle83}}
  Algorytm poprawnie wyznacza dolne przecięcie półpłaszczyzn.
\end{lemat}

Rozważmy rysunek~\ref{img:containment2}, na którym przedstawiono zbiór
,,dolnych'' półpłaszczyzn. Możemy zauważyć, że półpłaszczyzna zadana
przez prostą $l_o$ jest nadmiarowa przy wyznaczeniu przecięcia $l \cap
l_p \cap l_o \cap l_i$. Jeżeli wyeliminujemy półpłaszczyzny
nadmiarowe, możemy w prosty sposób wyznaczyć przecięcie.

Dzięki temu, że nachylenie kolejnych półpłaszczyzn jest monotoniczne
(w przypadku ,,dolnego'' zbioru --- nierosnące), wystarczy, że
wyznaczymy punkty przecięć kolejnych pod tym względem półpłaszczyzn,
uzyskując w ten sposób zbiór punktów będących wierzchołkami obszaru
przecięcia. Do określenia, czy dana półpłaszczyzna jest nadmiarowa,
opieramy się na następującym założeniu.

\begin{lemat}\emph{\cite{Brown78}}
  Niech będą dane półpłaszczyzny $H_1, H_2, H_3$. Niech $l_i$ oznacza
  prostą kierunkową półpłaszczyzny $H_i$, natomiast jako $slope(l_i)$
  oznaczmy współczynniki kierunkowy prostej $l_i$. Półpłaszczyzna
  $H_2$ jest nadmiarowa dla określenia ,,dolnego'' przecięcia $H_1
  \cap H_2 \cap H_3$ dokładnie wtedy, gdy oba poniższe warunki są
  spełnione:

  \begin{enumerate}
    \item Prosta $l_2$ leży poniżej punktu przecięcia prostych $l_1$ i
      $l_3$.
    \item $slope(l_1) < slope(l_2) < slope(l_3)$.
  \end{enumerate}
\end{lemat}

Zauważmy, że warunek drugi jest spełniony ze względu na wypukłość
wielokąta. Natomiast do sprawdzenia warunku pierwszego wystarczające
jest sprawdzenie, czy przecięcie obecnie rozważanej prostej z ostatnią
prostą na stosie leży na lewo od przecięcia przedostatniej i ostatniej
prostej ze stosu. Gdy proste są uporządkowane według współczynnika
nachylenia, powyższe warunki są równoważne --- jeżeli $l_2$ przecina
$l_1$ na lewo od przecięcia $l_1$ i $l_3$, przecięcie $l_1 \cap l_2$
musi znajdować się powyżej prostej $l_3$.

\section{Złożoność}
Pierwszą część algorytmu, czyli wyznaczenie wierzchołków krytycznych,
dzięki zastosowaniu techniki \emph{rotating calipers} można wykonać w
czasie liniowym. Wyznaczenie początkowej pozycji prostych równoległych
wymaga rozpatrzenia wszystkich kolejnych wierzchołków wielokątów $P$ i
$Q$, stąd wymagany wynosi $O(n + m)$. Po każdym obrocie prostych
równoległych jedna z nich jest styczna do krawędzi wspieranego
wielokąta. Za każdym razem, gdy prosta równoległa jest styczna do
krawędzi wielokąta $Q$, wyznaczany jest punkt krytyczny dla tej
krawędzi. Tak więc po wykonaniu pełnego obrotu wokół $P$ i $Q$
wyznaczymy wszystkie wierzchołki krytyczne. Przechylenie prostej
równoległej do krawędzi wielokąta wykonujemy w czasie $O(1)$, stąd
pełny obrót obydwu prostych równoległych wokół wspieranych wielokątów
wymaga czasu $O(n + m)$.  Łącznie złożoność czasowa tej części
algorytmu również wynosi $O(n + m)$.

Odległość punktu $c$, wyznaczającego umiejscowienie $P$ na
płaszczyźnie, od krawędzi wielokąta $Q$ możemy wyznaczyć w czasie
$O(1)$, jeżeli wyznaczyliśmy wierzchołek krytyczny dla tej krawędzi.

W~\cite{Brown78} Brown dowodzi równoważności między problemem
przecięcia dolnych półpłaszczyzn, a problemem otoczki wypukłej, który
może być rozwiązany w czasie liniowo-logarytmicznym (lub w czasie
liniowym, jeżeli punkty otoczki są posortowane według kolejności
krążenia wokół środka układu współrzędnych). Rozważając złożoność tej
części algorytmu możemy również spojrzeć na rozwiązanie w następujący
sposób: wyznaczając górne oraz dolne przecięcie rozważamy kolejne
proste wkładając lub zdejmując je ze stosu. Każda prosta może być
włożona oraz zdjęta ze stosu tylko raz, stąd pozbycie się nadmiarowych
półpłaszczyzn z obu zbiorów przecięć wymaga czasu $O(n + m)$. Takiej
samej złożoności czasowej wymaga wyznaczenie dolnego oraz górnego
przecięcia z uzyskanych półpłaszczyzn oraz połączenie uzyskanych
obszarów wypukłych.

Stąd złożoność czasowa całego algorytmu wynosi $O(n + m)$.

%%% Local Variables:
%%% mode: latex
%%% TeX-master: "masterthesis"
%%% TeX-engine: xetex
%%% End:

\summary{

}

%%% Local Variables:
%%% mode: latex
%%% TeX-master: "masterthesis"
%%% TeX-engine: xetex
%%% End:

% \appendix
% \chapter{Generowanie wielokątów wypukłych o zadanej liczbie
%   wierzchołków}

\bibliographystyle{unsrt}
\bibliography{literatura}

%%% Local Variables:
%%% mode: latex
%%% TeX-master: "masterthesis"
%%% TeX-engine: xetex
%%% End:

\listoftables
\listoffigures
\listofalgorithms

\addcontentsline{toc}{chapter}{\listalgorithmname}

\oswiadczenie

%%% Local Variables:
%%% mode: latex
%%% TeX-master: "masterthesis"
%%% TeX-engine: xetex
%%% End:


\end{document}

%%% Local Variables:
%%% coding: utf-8
%%% mode: latex
%%% TeX-master: t
%%% TeX-engine: xetex
%%% End:
