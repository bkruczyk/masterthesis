\chapter{Lokalizacja punktu}
W sytuacji gdy płaszczyznę dzielimy na dwa obszary, z których jeden
jest nieograniczony, drugi natomiast to wielokąt $P$ (na skutek
podziału płaszczyzny na dwie części oraz na podstawie twierdzenia
Jordana o krzywej dla wielokątów wiemy, że $P$ jest prosty) możemy
przedstawić definicję problemu:

\begin{problem}[Lokalizacja punktu]
  Jeśli dany jest wielokąt prosty $P$ i punkt $z$, sprawdzić czy punkt $z$
  należy do wnętrza $P$.
\end{problem}

Problem ten można rozwiązać w czasie $O(n)$ bez przetwarzania
wstępnego. Przypuśćmy, że istnieje prosta pozioma $l$ przechodząca
przez punkt $z$. Musimy teraz rozważyć teraz kilka przypadków:

\begin{itemize}
\item Jeśli $l$ nie przecina $P$ to $z$ leży na zewnątrz wielokąta.
  \begin{figure}[htp]
    \centering
    \caption{(Rysunek) Przypadek 1.}
  \end{figure}
\item W przypadku gdy prosta $l$ przecina $P$ i nie przechodzi przez
  żaden z wierzchołków $P$, jako $p_l$ będziemy oznaczać liczbę
  przecięć $l$ z wielokątem $P$. Wiemy, że lewy koniec $l$ leży na
  zawnątrz $P$, więc przesuwając się po prostej $l$ w prawo w kierunku
  $z$ i przecinając brzeg wielokąta i przechodzimy do jego
  wnętrza. Przy następnym przecięciu z brzegiem $P$ znowu znajdziemy
  się na zewnątrz itd. Widzimy stąd, że $z$ jest zewnętrzny dokładnie
  wtedy gdy $p_l$ jest parzyste.
  \begin{figure}[htp]
    \centering
    \caption{(Rysunek) Przypadek 2.}
  \end{figure}
\item Specjalnym przypadkiem jest ten w którym $l$ przechodzi przez
  jeden lub dwa wierzchołeki. W algorytmie rozważamy przecięcia $l$ z
  krawędziami $P$, więc musimy się zabezpieczyć przed sytuacją w
  której policzylibyśmy przecięcie brzegu $P$ dwukrotnie. Stosujemy tu
  zasadę, że $p_l$ nie zostanie zwiększone dla krawędzi której jeden z
  wierzchołków znajduję się powyżej prostej $l$.
  \begin{figure}[htp]
    \centering
    \caption{(Rysunek) Przypadek 3.}
  \end{figure}
\end{itemize}

Ze względu na to, że każdą krawędź $P$ sprawdzamy tylko raz pod względem
przecięcia z $l$ to mamy do czynienia ze złożonością liniową zależną od
liczby krawędzi wielokąta.

Algorytm dla wielokąta wypukłego wymaga pewnego przetwarzania
wstępnego, ale jest użyteczny w przypadku wykonowyania wielokrotnych
zapytań o przynależność punktu do danego wielokąta. Weźmy punkt $q$
leżący wewnątrz wielokąta wypukłego $P$, może to być np.\ jego środek
wyznaczony przez trzy wierzchołki, a nastąpnie poprowadźmy $N$
półprostych z $q$ przechodzących przez wierzchołki $P$. Płaszczyzna
wraz z wielokątem zostanie podzielona na $N$ części, które nazwiemy
klinami, z których każdy jest podzielony przez krawędź $P$ na dwie
częsci --- jest to wnętrze i zewnętrze wielokąta.

\begin{figure}[htp]
  \centering
  \caption{(Rysunek)}
\end{figure}

Wyszukiwanie położenia danego punktu $z$ składa się z dwóch
etapów. Najpierw określany jest klin w którym leży $z$. Można to
zrobić w czasie $O(\log n)$ używając wyszukiwania binarnego. Punkt $z$
będzie znajdował się pomiędzy promieniami $p_i$ i $p_{i+n}$
podzielonej płaszczyzny wtedy i tylko wtedy gdy kąt $(zqp_i)$ będzie
lewoskrętny, a kąt $(zqp_{i+n})$ prawoskrętny. W następnych krokach
stopniowo zawężamy nasz obszar poszukiwań, do czasu aż odnajdziemy
właściwy klin. Następnie określamy czy $z$ należy do wnętrza lub
zewnętrza znalezionego klinu. Jeśli kąt $(p_{i}p_{i+1}z)$ jest
lewoskętny to $z$ jest wewnętrzny. Określenie skrętności kąta jest
operacją o czasie $O(1)$, więc asymptotyczna złożoność tej części
algorytmu jest logarytmiczna zależna od $N$.

Zastosowane tutaj przeszukiwanie binarne jest możliwe dzięki temu, że
wierzchołki wielokąta wypukłego występują w kolejności kątowej lub
inaczej mówiąc w kolejności krążenia wokół punktu $q$. Przetwarzanie
wstępne dla wielokąta $P$ na które składa się wyznaczenie punktu $q$
oraz umieszczenie wielokąta w strukturze danych wspierającej
wyszukiwanie binarne, może zostać wykonane w czasie $O(n)$.

%%% Local Variables:
%%% mode: latex
%%% TeX-master: "masterthesis"
%%% TeX-engine: xetex
%%% End: