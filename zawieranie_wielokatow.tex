\chapter{Zawieranie wielokątów}
Ogólny problem zawierania wielokątów formułujemy następująco:

\begin{problem}[Ogólny problem zawierania wielokątów]
  Dla dwóch, prostych wielokątów $P$ i $Q$, stwierdzić czy $P$ może
  być zawarty w $Q$, i jeżeli tak, to podać umiejscowienie $P$, które
  spełnia zawieranie.
\end{problem}

\begin{figure}[htp]
  \caption{Zawieranie}
\end{figure}

W problemie ogólnym, dopuszczamy translacje i obroty zawieranego
wielokąta tak by ,,zmieścił się'' on wielokącie
zawierającym. Rozwiązanie tego problemu w czasie $O(pq^2)$ przedstawił
B. Chazelle[]. W swojej pracy Chazelle przedstawia także rozwiązanie
dla uproszczonego problemu w której wielokąty $P$ i $Q$ są wypukłe, a
dozwolonym działaniem na $P$ jest jedynie translacja.

Niech będzie dany wielokąt wypukły $P = (p_1, p_2, \ldots, p_n)$ oraz
wielokąt wypukły $Q = (q_1, q_2, \ldots, q_m)$. Zakładamy, że żadna
para krawędzi $P$ i $Q$ nie jest współliniowa. Niech $H_i$ dla $i = 1,
\ldots, m$ dla będzie i-tą półpłaszczyzną wyznaczoną przez prostą
zadaną punktami $q_i, q_{i+1}$ zawierającą wielokąt $Q$. Niech punkt
$c$ będzie dowolnym punktem należącym do $P$, może być to na przykład
środek jego masy. Punkt ten będzie wyznaczał umiejscowienie $P$ na
płaszczyźnie.

Zauważmy, że ,,przesuwając'' $P$ tak, by punkt $c$ znajdował się w jak
najmniejszej odległości od $H_i$, istnieje jeden wierzchołek $p_j \in
P$ styczny z prostą $L(q_i, q_{i+1})$ wyznaczającą półpłaszczyznę
$H_i$. O takim wierzchołku $p_j$ mówimy, że jest \emph{krytyczny} dla
krawędzi $(q_i, q_{i+1}) \in Q$. Odległość punktu $c$ od
półpłaszczyzny $H_i$ (i tym samym odległość od $p_j$) oznaczmy jako
$d_i$.

Możemy również zauważyć, że warunek zawarcia $P$ w $Q$ jest
równoznaczny z zawieraniem $P$ przecięciu półpłaszczyzn:
\[
       \bigcap_{1 \leq i \leq m} H_i
\]

Rozważmy półpłaszczyznę $H_i'$ wyznaczoną przez prostą
$L(q_i,q_{i+1})'$, równoległą do $L(q_i, q_{i+1})$ i przechodzącą
przez punkt $c$.  $L(q_i,q_{i+1})$ możemy łatwo wyznaczyć posługując
się odległością punktu $c$ od wierzchołka $p_j$, oznaczoną wcześniej
jako $d_i$. Poprzedni warunek na zawierania wielokąta możemy
przekształcić na zawieranie punktu $c$ w przecięciu półpłaszczyzn
$H_i'$. Innymi słowy wielokąt $P$ zawiera się w $Q$ dokładnie wtedy
gdy punkt $c$ należy do przecięcia półpłaszczyzn $H_i'$. Co więcej,
wspomniane przecięcie zawiera wszystkie możliwe położenia $c$
spełniające zawieranie $P$ w $Q$. Algorytm dla problemu możemy
przedstawić następująco:

\begin{figure}[htp]
\begin{algorithmic}[1]
\Procedure{Convex Polygon Containment}{}

\State \textbf{dla każdej} krawędzi $(q_i, q_{i+1}) \in Q$
\State \hspace{\algorithmicindent} wyznacz wierzchołek krytyczny $p_j \in P$
\State \hspace{\algorithmicindent} wyznacz półpłaszczyznę $H_i'$

\State $placement \gets $ przecięcie półpłaszczyzn $H_i$

\EndProcedure
\end{algorithmic}
\end{figure}

Do znalezienia wierzchołków krytycznych możemy się posłużyć
przedstawioną w rozdziale \ref{chap:calipers} metodą \emph{rotating
  calipers}. Zauważmy, że dana krawędź $(q_i, q_{i+1})$ wraz z
odpowiadającym jej wierzchołkiem krytycznym $p_j$ tworzą \emph{parę
  kopodalną}. Rozważmy sytuację przedstawioną na rysunku (). Niech
początkowymi liniami wsparcia dla $P$ i $Q$ będą proste równoległe do
osi $O_x$, styczne do wierzchołka o najmniejszej wartości współrzędnej
$x$ z $P$ i $Q$ odpowiednio. Niech obydwie proste wspierające będą
skierowane w tym samym kierunku w ten sposób, żeby wspierany wielokąt
znajdował się po lewej stronie prostej. Analogicznie jak w rozdziale
\ref{chap:calipers} zaczynamy obracać równoległe proste wspierające. Przy
każdym obrocie suwmiarek po którym prosta wspierająca $L_{SQ}$ jest
styczna z krawędzią $Q$, prosta wspierająca $L_{SP}$ jest styczna z
punktem krytycznym dla tej krawędzi. Następnie dla każdej krawędzi $Q$
i odpowiadającemu jej punktowi krytycznemu wyznaczamy półpłaszczyznę
$H_i'$, tak jak zostało to opisane wcześniej.

Na koniec pozostaje wyznaczenie przecięcia półpłaszczyzn $H_1' \cap
\ldots \cap H_m'$. Uzyskane półpłaszczyzny podzielmy dwa zbiory: dolny
i górny. Niech w dolnym zbiorze znajdą się półpłaszczyzny wyznaczone
przez ,,dolną'' część wielokąta $Q$. Zacznijmy rozpatrywać kolejne
półpłaszczyzny z dolnego zbioru pod względem nierosnącego nachylenia
prostej $l_i$ definiującej półpłaszczyznę do osi $O_x$. Włóżmy dwie
pierwsze proste na stos. Będziemy oznaczmać ostatnią prostą na stosie
jako $l_o$, a przedostatnią jako $l_p$. Po rozpatrzeniu każdej
kolejnej półpłaszczyzny będziemy zachowywać następujący niezmiennik:
na stosie znajdują się te proste z ciągu $l_1, \ldots, l_i$, które
stanowią brzeg zbioru $H_1' \cap \ldots \cap H_i'$. Postępujemy
zgodnie z regułą, że jeżeli przecięcie obecnie rozważanej prostej
$l_i$ z ostatnia prostą na stosie $l_o$ leży na lewo od przecięcia
$l_o$ i przedostatnią prostą ze stosu $l_p$ to zdejmujemy $l_o$ ze
stosu i wkładamy na stos prostą $l_i$. Analogicznie postępujemy z
zbiorem ,,górnych'' półpłaszczyzn. W ostatnim kroku łączymy uzyskane
dwa obszary wypukłe, poprzez wyznaczenie przecięcia par skrajnych
półpłaszczyzn z ,,dolnej'' i ,,górnej'' otoczki.

\section{Poprawność}
Poprawność warunku zawierania $P$ w $Q$ jako warunku zawierania $P$ w
przecięciu półpłaszczyzn wyznaczonych przez krawędzie $Q$ opiera się
na następującym założeniu:

\begin{lemat}[Przecięcie półpłaszczyzn]
  Przecięcie $n$ półpłaszczyzn jest wielokątem wypukłym.
\end{lemat}

Co wynika bezpośrednio z poniższych:

\begin{twierdzenie}
  Półpłaszczyzna jest zbiorem wypukłym [].
\end{twierdzenie}

\begin{twierdzenie}
  Przecięcie zbiorów wypukłych jest zbiorem wypukłym [].
\end{twierdzenie}

Drugim założeniem na którym opiera się algorytm jest:

\begin{lemat}
  Dla każdej krawędzi $Q$ istnieje wierzchołek krytyczny z $P$.
\end{lemat}

Zakładając, że żadne trzy wierzchołki $P$ nie są współliniowe to z
wypukłości $P$ wynika, że istnieją co najwyżej dwa najbardziej
zbliżone do krawędzi $q_i,q_{i+1}$ wierzchołki. Ma to miejsce w
przypadku, gdy krawędź $q_i,q_{i+1}$ jest równoległa do krawędzi $P$
zawierającej obydwa wierzchołki krytyczne. W przeciwnym przypadku
istnieje dokładniej jeden wierzchołek krytyczny.

\begin{lemat}
  Każdy obrót prostej równoległej wokół $Q$ wyznacza punkt krytyczny
  dla bieżącej krawędzi $Q$.
\end{lemat}

\begin{lemat}
  Algorytm \textsc{Lower Bound Intersection} poprawnie wyznacza
  ,,dolne'' przecięcie półpłaszczyzn.
\end{lemat}

Rozważmy rysunek. na którym przedstawiono zbiór ,,dolnych''
półpłaszczyzn. Możemy zauważyć, że półpłaszczyzna $H_2$ jest
\emph{nadmiarowa} przy wyznaczeniu przecięcia $H_1 \cap H_2 \cap
H_3$. Jeżeli wyeliminujemy ze półpłaszczyzny nadmiarowe, możemy w
prosty sposób wyznaczyć przecięcie. Dzięki temu, że nachylenie
kolejnych półpłaszczyzn względem osi $O_x$ jest monotoniczne (w
przypadku ,,dolnego zbioru nierosnące), wystarczy, że wyznaczymy
punkty przecięć kolejnych pod względem nachylenia półpłaszczyzn
uzyskując w ten sposób zbiór punktów będących wierzchołkami obszaru
przecięcia. Do określenia czy dana półpłaszczyzna jest nadmiarowa
opieramy się na następującym lemacie:

\begin{lemat}[Kevin Brown 1978]
  Niech będą dane półpłaszczyzny $H_1, H_2, H_3$. Niech $l_i$ oznacza
  prostą kierunkową półpłaszczyzny $H_i$, natomiast jako $slope(l_i)$
  oznaczmy kąt nachylenia prostej $l_i$ do osi $O_x$. Półpłaszczyzna
  $H_2$ jest \emph{nadmiarowa} dla określenia ,,dolnego'' przecięcia
  $H_1 \cap H_2 \cap H_3$ dokładnie wtedy gdy oba poniższe warunki są
  spełnione:

  \begin{enumerate}
    \item Prosta $l_2$ leży poniżej punktu przecięcia prostych $l_1$ i
      $l_3$
    \item $slope(l_1) < slope(l_2) < slope(l_3)$
  \end{enumerate}
\end{lemat}

Zauważmy, że warunek drugi jest spełniony ze względu na wypukłość
wielokąta. Natomiast do sprawdzenia warunku pierwszego wystarczające
jest sprawdzenie czy przecięcie obecnie rozważanej prostej z ostatnią
prostą na stosie leży na lewo od przecięcia przedostatniej i ostatniej
prostej ze stosu. Gdy proste są uporządkowane według współczynnika
nachylenia powyższe warunki są równoważne --- jeżeli $l_2$ przecina
$l_1$ na lewo od przecięcia $l_1$ i $l_3$, przecięcie $l_1 \cap l_2$
musi znajdować się powyżej prostej $l_3$.

\section{Złożoność}
Pierwsza część algorytmu, czyli wyznaczenie wierzchołków krytycznych,
dzięki zastosowaniu techniki \emph{rotating calipers} można wykonać w
czasie liniowym. Wyznaczenie początkowej pozycji prostych równoległych
wymaga rozpatrzenia wszystkich kolejnych wierzchołków $P$ i $Q$ stąd
wymagany wynosi $O(n + m)$. Po każdym obrocie prostych równoległych
jedna z nich jest styczna do krawędzi wspieranego wielokąta. Za każdym
razem gdy, prosta równoległa jest styczna do krawędzi $Q$, wyznaczany
jest punkt krytyczny dla tej krawędzi. Tak więc po wykonaniu pełnego
obrotu wokół $P$ i $Q$ wyznaczymy wszystkie wierzchołki
krytyczne. ,,Przechylenie'' prostej równoległej do krawędzi wielokąta
wykonujemy w czasie $O(1)$, stąd pełny obrót obydwu prostych
równoległych wokół wspieranych wielokątów wymaga czasu $O(n + m)$.
Łącznie złożoność czasowa tej części algorytmu również wynosi $O(n +
m)$.

Odległość punktu $c$, wyznaczającego umiejscowienie $P$ na
płaszczyźnie, od krawędzi $Q$ możemy wyznaczyć w czasie $O(1)$, jeżeli
wyznaczyliśmy wierzchołek krytyczny dla tej krawędzi.

W [], K. Brown dowodzi równoważności między problemem przecięcia
,,dolnych'' półpłaszczyzn, a problemem otoczki wypukłej, który może
być rozwiązany w czasie optymalnym $O(n)$. Rozważając złożoność tej
części algorytmu możemy również spojrzeć w na rozwiązanie w
następujący sposób: wyznaczając ,,górne'' oraz ,,dolne'' przecięcie
rozważamy kolejne proste wkładając lub zdejmując je ze stosu. Każda
prosta może być włożona, oraz zdjęta ze stosu tylko raz, stąd pozbycie
się nadmiarowych półpłaszczyzn z obu zbiorów przecięć wymaga czasu
$O(n + m)$. Takiej samej złożoności czasowej wymaga wyznaczenie
dolnego oraz górnego przecięcia z uzyskanych półpłaszczyzn oraz
połączenie uzyskanych obszarów wypukłych.

Stąd złożoność czasowa całego algorytmu wynosi $O(n + m)$.

%%% Local Variables:
%%% mode: latex
%%% TeX-master: "masterthesis"
%%% TeX-engine: xetex
%%% End:
