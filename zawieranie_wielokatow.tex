\chapter{Zawieranie wielokątów}
Ogólny problem zawierania wielokątów formułujemy następująco:

\begin{problem}[Ogólny problem zawierania wielokątów]
  Dla dwóch, prostych wielokątów $P$ i $Q$, stwierdzić czy $P$ może
  być zawarty w $Q$, i jeżeli tak, to podać umiejscowienie $P$, które
  spełnia zawieranie.
\end{problem}

\begin{figure}[htp]
  \caption{Zawieranie}
\end{figure}

W problemie ogólnym, dopuszczamy translacje i obroty zawieranego
wielokąta tak by ,,zmieścił się'' on wielokącie
zawierającym. Rozwiązanie tego problemu w czasie $O(pq^2)$ przedstawił
B. Chazelle[]. W swojej pracy Chazelle przedstawia także rozwiązanie
dla uproszczonego problemu w której wielokąty $P$ i $Q$ są wypukłe, a
dozwolonym działaniem na $P$ jest jedynie translacja.

Niech będzie dany wielokąt wypukły $P = (p_1, p_2, \ldots, p_n)$ oraz
wielokąt wypukły $Q = (q_1, q_2, \ldots, q_m)$. Zakładamy, że żadna
para krawędzi $P$ i $Q$ nie jest współliniowa. Niech $H_i$ dla $i = 1,
\ldots, m$ dla będzie i-tą półpłaszczyzną wyznaczoną przez prostą
zadaną punktami $q_i, q_{i+1}$ zawierającą wielokąt $Q$. Niech punkt
$c$ będzie dowolnym punktem należącym do $P$, może być to na przykład
środek jego masy. Punkt ten będzie wyznaczał umiejscowienie $P$ na
płaszczyźnie.

Zauważmy, że ,,przesuwając'' $P$ tak, by punkt $c$ znajdował się w jak
najmniejszej odległości od $H_i$, istnieje jeden wierzchołek $p_j \in
P$ styczny z prostą $L(q_i, q_{i+1})$ wyznaczającą półpłaszczyznę
$H_i$. O takim wierzchołku $p_j$ mówimy, że jest \emph{krytyczny} dla
krawędzi $(q_i, q_{i+1}) \in Q$. Odległość punktu $c$ od
półpłaszczyzny $H_i$ (i tym samym odległość od $p_j$) oznaczmy jako
$d_i$.

Możemy również zauważyć, że warunek zawarcia $P$ w $Q$ jest
równoznaczny z zawieraniem $P$ przecięciu półpłaszczyzn:
\[
       \bigcap_{1 \leq i \leq m} H_i
\]

Rozważmy półpłaszczyznę $H_i'$ wyznaczoną przez równoległą do $L(q_i,
q_{i+1})$ linię przechodzącą przez punkt $c$. Poprzedni warunek możemy
przekształcić na zawieranie punktu $c$ w przecięciu półpłaszczyzn
$H_i'$. Innymi słowy wielokąt $P$ zawiera się w $Q$ dokładnie wtedy
gdy punkt $c$ należy do przecięcia półpłaszczyzn $H_i'$. Co więcej,
wspomniane przecięcie zawiera wszystkie możliwe położenia $c$
spełniające zawieranie $P$ w $Q$.

Algorytm możemy przedstawić w trzech krokach.

\begin{figure}[htp]
\begin{algorithmic}[1]
\Procedure{Polygon Containment}{}
\EndProcedure
\end{algorithmic}
\end{figure}

%%% Local Variables:
%%% mode: latex
%%% TeX-master: "masterthesis"
%%% TeX-engine: xetex
%%% End:
