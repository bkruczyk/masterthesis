\chapter{Przecięcie wielokątów}
Problem znalezienia przecięcia dla wielokątów wypukłych zdefiniujmy
następująco:

\begin{problem}[Znajdowanie przecięcia wielokątów]
  Gdy dane są dwa wielokąty wypukłe $P$ i $Q$, znaleźć ich
  \emph{przecięcie} tj.\ obszar należący jednocześnie do $P$ i do $Q$.
\end{problem}

W niniejszym rozdziale do wielokąta będziemy zaliczać jego wnętrze
wraz z brzegiem.

Oznaczmy jako $N$ liczbę wierzchołków $P$, a jako $M$ liczbę
wierzchołków $Q$. Stosując algorytm naiwny polegający na sprawdzeniu
przecięcia każdej krawędzi $P$ z każdą krawędzią $Q$ można uzyskać
punkty wszystkich przecięć w czasie $O(NM)$.

\section{Metoda warstwowa}
Znaną metodą rozwiązywania problemów w algorytmice jest technika
\emph{dziel i rządź}, polegająca na podziale na mniejsze części,
znajdowaniu rozwiązań częściowych, a następnie łączeniu ich w końcowy
rezultat.

Metodę wykorzystującą podział płaszczyzny zaproponowali w 1976 Shamos
i Hoey. Przez każdy wierzchołek $P$ i $Q$ prowadzimy prostą równoległą
do osi $y$, tworząc w ten sposób $N+M$ warstw. Znalezienie przecięć
krawędzi obydwu wielokątów w każdej warstwie pozwoli nam na proste
wyznaczenie obszaru przecięcia tych wielokątów.

Należy zauważyć, że w każdej warstwie proste ją ograniczające wraz z
krawędziami wielokąta tworzą czworobok lub gdy jest to wierzchołek
skrajny na osi $x$ --- trójkąt. W pojedynczej warstwie przecięcie
dwóch czworoboków możemy wyznaczyć w czasie stałym. Z twierdzenia
\ref{thm:egdecollision} wiemy, że w każdej warstwie znajdziemy nie
więcej niż dwa przecięcia.

\begin{twierdzenie}
\label{thm:egdecollision}
  Jeżeli dwa wielokąty wypukłe $P$ i $Q$ przecinają się, to każda z
  krawędzi $P$ przecina $Q$ co najwyżej dwa razy.
\end{twierdzenie}

Następnie w czasie liniowym łączymy uzyskane kawałki i usuwamy
nadmiarowe wierzchołki powstałe przy podziale płaszczyzny. Początkowy
podział płaszczyzny również możemy przeprowadzić w czasie liniowym,
tak więc złożoność czasowa całego algorytmu wynosi $O(n)$.

%%% Local Variables:
%%% mode: latex
%%% TeX-master: "masterthesis"
%%% TeX-engine: xetex
%%% End:
