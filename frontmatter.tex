\begin{abstract}

  Celem niniejszej pracy jest przedstawienie szeregu problemów
  geometrycznych dotyczących wielokątów wypukłych oraz ich rozwiązań
  korzystających z własności wypukłości, prowadząc do podania
  algorytmów działających w czasie liniowym. Opis rozwiązań zawiera
  analizę działania wraz ze złożonością czasową i dowodem poprawności
  wybranych algorytmów.

  Rozdział pierwszy zawiera podstawowe definicje i pojęcia
  geometryczne związane z tematem pracy oraz przedstawienie notacji
  złożoności analizowanych algorytmów.  W rozdziale drugim
  przedstawiono rozwiązania problemu lokalizacji punktu dla wielokąta
  prostego oraz dla wielokąta wypukłego z wykorzystaniem techniki
  podziału wielokąta na kliny.  Rozdział trzeci opisuje rozwiązanie
  problemu wyznaczenia średnicy wielokąta wypukłego wykorzystując do
  tego celu pojęcie par antypodycznych, które zostało szerzej
  wykorzystane w rozdziale piątym.  Rozdział czwarty zawiera
  przedstawienie trzech algorytmów rozwiązujących problem przecięcia
  wielokątów wypukłych: wykorzystujący technikę zamiatania płaszczyzny
  algorytm Shamosa-Hoeya, algorytm O'Rourke i in., który korzysta z
  pojęcia sierpów, oraz algorytm Toussainta używający do tego celu
  pojęcia kieszeni.  W rozdziale piątym przedstawiono algorytmy z
  wykorzystaniem techniki \emph{rotating calipers} opierającej się na
  przedstawionym w rozdziale drugim pojęciu par
  antypodycznych. Opisano alternatywne rozwiązanie dla problemu
  wyznaczania średnicy, a także przedstawiono rozwiązania problemów
  najmniejszego prostokąta zawierającego wielokąt wypukły, wyznaczania
  największej odległości między wielokątami wypukłymi oraz łączenia
  otoczek wypukłych. Rozdział szósty opisuje problem zawierania
  wielokątów oraz przedstawia rozwiązanie tego problemu w warunkach,
  gdy wielokąty są wypukłe, a jedynym możliwym działaniem na
  wielokącie zawieranym jest translacja.

\end{abstract}

\keywords{
  wielokąty wypukłe,
  lokalizacja punktu,
  średnica wielokąta,
  przecięcie wielokątów,
  rotating calipers,
  najmniejszy prostokąt zawierający,
  największa odległość,
  łączenie otoczek wypukłych,
  zawieranie wielokątów
}

\maketitle

%%% Local Variables:
%%% mode: latex
%%% TeX-master: "masterthesis"
%%% TeX-engine: xetex
%%% End:
