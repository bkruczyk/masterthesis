\introduction{
  Problemy geometryczne, ze względu na swoje praktyczne zastosowania,
  są jednym z najstarszym przedmiotów rozważań algorytmicznych. Wraz z
  rozwojem technologi, narodziło się pojęcie geometrii obliczeniowej
  --- dyscypliny rozwiązującej problemy geometryczne z wykorzystaniem
  komputera w modelu obliczeń. Wiele z rozwiązań znalazło użytek w
  systemach informacji geograficznej, robotyce, grafice komputerowej
  czy też w narzędziach CAD/CAM wspomagających pracę projektantów i
  architektów.

  Okazuje się, że posiadanie przez wielokąt własności wypukłości
  pozwala na podanie szybszych lub prostszych rozwiązań tego samego
  problemu dla dowolnego wielokąta. W niniejszej pracy przedstawiono
  problemy dotyczące wielokątów wypukłych oraz algorytmy pozwalające
  na ich efektywne rozwiązanie.

  W rozdziale pierwszym podano podstawowe definicje i pojęcia
  geometryczne związane z tematem pracy oraz przedstawiono notację
  złożoności analizowanych algorytmów. Rozdział drugi oraz trzeci
  przedstawiają wykorzystanie wypukłości w rozwiązaniach o czasie
  liniowym problemów lokalizacji punktu oraz wyznaczania średnicy. W
  rozdziale czwartym zostają omówione trzy algorytmy rozwiązujące
  problem znajdowania przecięcia wielokątów wypukłych, który jest bazą
  dla takich problemów jak suma i różnica wielokątów. Rozdział piąty
  zawiera rozwiązania szeregu problemów z wykorzystaniem techniki
  ,,obracających się suwmiarek'' (ang.~\emph{rotating
    calipers}). Natomiast w rozdziale szóstym zostaje omówiony problem
  zawierania jednego wielokąta w drugim oraz znalezienia punktu
  realizującego takie zawieranie.
 }

%%% Local Variables:
%%% mode: latex
%%% TeX-master: "masterthesis"
%%% TeX-engine: xetex
%%% End:
