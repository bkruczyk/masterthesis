\introduction{
% geometria obliczeniowa i problemy geometryczne, zastosowania
% wykorzystanie wypuklosci w problemach geometria
% niniejsza praca

  Problemy geometryczne, ze względu na swoje praktyczne zastosowania,
  są jednym z najstarszym przedmiotów rozważań algorytmicznych. Wraz z
  rozwojem technologi, narodziło się pojęcie geometrii obliczeniowej
  --- dyscypliny rozwiązującej problemy geometryczne z wykorzystaniem
  komputerów w modelu obliczeń. Wiele z rozwiązań znalazło użytek w
  systemach informacji geograficznej, robotyce, grafice komputerowej
  czy też w narzędziach CAD/CAM wspomagających pracę projektantów i
  architektów.

  Okazuje się, że posiadanie przez wielokąt własności wypukłości,
  pozwala na podanie szybszych lub prostszych rozwiązań takiego samego
  problemu dla dowolnego wielokąta. Celem niniejszej pracy jest
  przedstawienie szeregu problemów dotyczących wielokątów wypukłych
  oraz ich rozwiązań.

  W rozdziale pierwszym...

  W rozdziale drugim...

  W rozdziale
}

%%% Local Variables:
%%% mode: latex
%%% TeX-master: "masterthesis"
%%% TeX-engine: xetex
%%% End:
